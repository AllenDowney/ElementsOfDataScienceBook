\chapter{Political Alignment and
Outlook}\label{political-alignment-and-outlook}

In the previous chapter, we used data from the General Social Survey
(GSS) to plot changes in political alignment over time. In this
notebook, we'll explore the relationship between political alignment and
respondents' beliefs about themselves and other people. We'll use the
following variables from the GSS dataset:

\begin{itemize}
\item
  \passthrough{\lstinline!happy!}: Taken all together, how would you say
  things are these days--would you say that you are very happy, pretty
  happy, or not too happy?
\item
  \passthrough{\lstinline!trust!}: Generally speaking, would you say
  that most people can be trusted or that you can't be too careful in
  dealing with people?
\item
  \passthrough{\lstinline!helpful!}: Would you say that most of the time
  people try to be helpful, or that they are mostly just looking out for
  themselves?
\item
  \passthrough{\lstinline!fair!}: Do you think most people would try to
  take advantage of you if they got a chance, or would they try to be
  fair?
\end{itemize}

We'll start with the last question; then as an exercise you can look at
one of the others. Here's the plan:

\begin{enumerate}
\def\labelenumi{\arabic{enumi}.}
\item
  First we'll use \passthrough{\lstinline!groupby!} to compare the
  average response between groups and plot the average as a function of
  time.
\item
  We'll use the Pandas function \passthrough{\lstinline!pivot table!} to
  compute the average response within each group as a function of time.
\item
  Finally, we'll use resampling to see whether the features we see in
  the results might be due to randomness, or whether they are likely to
  reflect actual changes in the works.
\end{enumerate}

We'll use an extract of the data that I have cleaned and resampled to
correct for stratified sampling. Instructions for downloading the file
are in the notebook for this chapter. It contains three resamplings --
we'll use the first, \passthrough{\lstinline!gss0!}, to get started. At
the end of the chapter, we'll use the other two as well.

\begin{lstlisting}[language=Python,style=source]
datafile = "gss_pacs_resampled.hdf"
gss = pd.read_hdf(datafile, "gss0")
gss.shape
\end{lstlisting}

\begin{lstlisting}[style=output]
(72390, 207)
\end{lstlisting}

\section{Are People Fair?}\label{are-people-fair}

In the GSS data, the variable \passthrough{\lstinline!fair!} contains
responses to this question:

\begin{quote}
Do you think most people would try to take advantage of you if they got
a chance, or would they try to be fair?
\end{quote}

The possible responses are:

\begin{longtable}[]{@{}ll@{}}
\midrule\noalign{}
Code & Response \\
\midrule\noalign{}
\endhead
\midrule\noalign{}
\endlastfoot
1 & Take advantage \\
2 & Fair \\
3 & Depends \\
\end{longtable}

As always, we start by looking at the distribution of responses, that
is, how many people give each response:

\begin{lstlisting}[language=Python,style=source]
values(gss["fair"])
\end{lstlisting}

\begin{lstlisting}[style=output]
fair
1.0    16089
2.0    23417
3.0     2897
NaN    29987
Name: count, dtype: int64
\end{lstlisting}

The plurality think people try to be fair (2), but a substantial
minority think people would take advantage (1). There are also a number
of NaNs, mostly respondents who were not asked this question.

\begin{lstlisting}[language=Python,style=source]
gss["fair"].isna().sum()
\end{lstlisting}

\begin{lstlisting}[style=output]
29987
\end{lstlisting}

To count the number of people who chose option
\passthrough{\lstinline!2!}, ``people try to be fair'', we'll use a
dictionary to recode option \passthrough{\lstinline!2!} as
\passthrough{\lstinline!1!} and the other options as
\passthrough{\lstinline!0!}.

\begin{lstlisting}[language=Python,style=source]
recode_fair = {1: 0, 2: 1, 3: 0}
\end{lstlisting}

As an alternative, we could include option \passthrough{\lstinline!3!},
``depends'', by replacing it with \passthrough{\lstinline!1!}, or give
it less weight by replacing it with an intermediate value like
\passthrough{\lstinline!0.5!}. We can use
\passthrough{\lstinline!replace!} to recode the values and store the
result as a new column in the \passthrough{\lstinline!DataFrame!}.

\begin{lstlisting}[language=Python,style=source]
gss["fair2"] = gss["fair"].replace(recode_fair)
\end{lstlisting}

And we'll use \passthrough{\lstinline!values!} to make sure it worked.

\begin{lstlisting}[language=Python,style=source]
values(gss["fair2"])
\end{lstlisting}

\begin{lstlisting}[style=output]
fair2
0.0    18986
1.0    23417
NaN    29987
Name: count, dtype: int64
\end{lstlisting}

Now let's see how the responses have changed over time.

\section{Fairness Over Time}\label{fairness-over-time}

As we saw in the previous chapter, we can use
\passthrough{\lstinline!groupby!} to group responses by year.

\begin{lstlisting}[language=Python,style=source]
gss_by_year = gss.groupby("year")
\end{lstlisting}

From the result we can select \passthrough{\lstinline!fair2!} and
compute the mean.

\begin{lstlisting}[language=Python,style=source]
fair_by_year = gss_by_year["fair2"].mean()
\end{lstlisting}

Here's the result, which shows the fraction of people who say people try
to be fair, plotted over time. As in the previous chapter, we plot the
data points themselves with circles and a local regression model as a
line.

\begin{lstlisting}[language=Python,style=source]
plot_series_lowess(fair_by_year, "C1")

decorate(
    xlabel="Year",
    ylabel="Fraction saying yes",
    title="Would most people try to be fair?",
)
\end{lstlisting}

\begin{center}
\includegraphics[width=4in]{chapters/15_outlook_files/15_outlook_29_0.png}
\end{center}

Sadly, it looks like faith in humanity has declined, at least by this
measure. Let's see what this trend looks like if we group the
respondents by political alignment.

\section{Political Views on a 3-point
Scale}\label{political-views-on-a-3-point-scale}

In the previous notebook, we looked at responses to
\passthrough{\lstinline!polviews!}, which asks about political
alignment. To make it easier to visualize groups, we'll lump the 7-point
scale into a 3-point scale.

\begin{lstlisting}[language=Python,style=source]
recode_polviews = {
    1: "Liberal",
    2: "Liberal",
    3: "Liberal",
    4: "Moderate",
    5: "Conservative",
    6: "Conservative",
    7: "Conservative",
}
\end{lstlisting}

We'll use \passthrough{\lstinline!replace!} again, and store the result
as a new column in the \passthrough{\lstinline!DataFrame!}.

\begin{lstlisting}[language=Python,style=source]
gss["polviews3"] = gss["polviews"].replace(recode_polviews)
\end{lstlisting}

With this scale, there are roughly the same number of people in each
group.

\section{Fairness by Group}\label{fairness-by-group}

Now let's see who thinks people are more fair, conservatives or
liberals. We'll group the respondents by
\passthrough{\lstinline!polviews3!}.

\begin{lstlisting}[language=Python,style=source]
by_polviews = gss.groupby("polviews3")
\end{lstlisting}

And compute the mean of \passthrough{\lstinline!fair2!} in each group.

\begin{lstlisting}[language=Python,style=source]
by_polviews["fair2"].mean()
\end{lstlisting}

\begin{lstlisting}[style=output]
polviews3
Conservative    0.577879
Liberal         0.550849
Moderate        0.537621
Name: fair2, dtype: float64
\end{lstlisting}

It looks like conservatives are a little more optimistic, in this sense,
than liberals and moderates. But this result is averaged over the last
50 years. Let's see how things have changed over time.

\section{Fairness over Time by Group}\label{fairness-over-time-by-group}

So far, we have grouped by \passthrough{\lstinline!polviews3!} and
computed the mean of \passthrough{\lstinline!fair2!} in each group. Then
we grouped by \passthrough{\lstinline!year!} and computed the mean of
\passthrough{\lstinline!fair2!} for each year. Now we'll group by
\passthrough{\lstinline!polviews3!} and \passthrough{\lstinline!year!},
and compute the mean of \passthrough{\lstinline!fair2!} in each group
over time.

We could do that computation ``by hand'' using the tools we already
have, but it is so common and useful that it has a name. It is called a
\textbf{pivot table}, and Pandas provides a function called
\passthrough{\lstinline!pivot\_table!} that computes it. It takes the
following arguments:

\begin{itemize}
\item
  \passthrough{\lstinline!values!}, which is the name of the variable we
  want to summarize: \passthrough{\lstinline!fair2!} in this example.
\item
  \passthrough{\lstinline!index!}, which is the name of the variable
  that will provide the row labels: \passthrough{\lstinline!year!} in
  this example.
\item
  \passthrough{\lstinline!columns!}, which is the name of the variable
  that will provide the column labels:
  \passthrough{\lstinline!polview3!} in this example.
\item
  \passthrough{\lstinline!aggfunc!}, which is the function used to
  ``aggregate'', or summarize, the values:
  \passthrough{\lstinline!mean!} in this example.
\end{itemize}

Here's how we run it.

\begin{lstlisting}[language=Python,style=source]
table = gss.pivot_table(
    values="fair2", index="year", columns="polviews3", aggfunc="mean"
)
\end{lstlisting}

The result is a \passthrough{\lstinline!DataFrame!} that has years
running down the rows and political alignment running across the
columns. Each entry in the table is the mean of
\passthrough{\lstinline!fair2!} for a given group in a given year.

\begin{lstlisting}[language=Python,style=source]
table.head()
\end{lstlisting}

\begin{tabular}{lrrr}
\midrule
polviews3 & Conservative & Liberal & Moderate \\
year &  &  &  \\
\midrule
1975 & 0.625616 & 0.617117 & 0.647280 \\
1976 & 0.631696 & 0.571782 & 0.612100 \\
1978 & 0.694915 & 0.659420 & 0.665455 \\
1980 & 0.600000 & 0.554945 & 0.640264 \\
1983 & 0.572438 & 0.585366 & 0.463492 \\
\midrule
\end{tabular}

Reading across the first row, we can see that in 1975, moderates were
slightly more optimistic than the other groups. Reading down the first
column, we can see that the estimated mean of
\passthrough{\lstinline!fair2!} among conservatives varies from year to
year. It is hard to tell looking at these numbers whether it is trending
up or down -- we can get a better view by plotting the results.

\section{Plotting the Results}\label{plotting-the-results}

Before we plot the results, I'll make a dictionary that maps from each
group to a color. Seaborn provide a palette called
\passthrough{\lstinline!muted!} that contains the colors we'll use.

\begin{lstlisting}[language=Python,style=source]
muted = sns.color_palette("muted", 5)
sns.palplot(muted)
\end{lstlisting}

\begin{center}
\includegraphics[width=4in]{chapters/15_outlook_files/15_outlook_49_0.png}
\end{center}

And here's the dictionary.

\begin{lstlisting}[language=Python,style=source]
color_map = {"Conservative": muted[3], "Moderate": muted[4], "Liberal": muted[0]}
\end{lstlisting}

Now we can plot the results.

\begin{lstlisting}[language=Python,style=source]
groups = ["Conservative", "Liberal", "Moderate"]
for group in groups:
    series = table[group]
    plot_series_lowess(series, color_map[group])

decorate(
    xlabel="Year",
    ylabel="Fraction saying yes",
    title="Would most people try to be fair?",
)
\end{lstlisting}

\begin{center}
\includegraphics[width=4in]{chapters/15_outlook_files/15_outlook_53_0.png}
\end{center}

The fraction of respondents who think people try to be fair has dropped
in all three groups, although liberals and moderates might have leveled
off. In 1975, liberals were the least optimistic group. In 2022, they
might be the most optimistic. But the responses are quite noisy, so we
should not be too confident about these conclusions.

We can get a sense of how reliable they are by running the resampling
process a few times and checking how much the results vary.

\section{Simulating Possible
Datasets}\label{simulating-possible-datasets}

The figures we have generated so far in this notebook are based on a
single resampling of the GSS data. Some of the features we see in these
figures might be due to random sampling rather than actual changes in
the world. By generating the same figures with different resampled
datasets, we can get a sense of how much variation there is due to
random sampling.

To make that easier, the following function contains the code from the
previous analysis all in one place.

\begin{lstlisting}[language=Python,style=source]
def plot_by_polviews(gss):
    """Plot mean response by polviews and year.

    gss: DataFrame
    """
    gss["polviews3"] = gss["polviews"].replace(recode_polviews)
    gss["fair2"] = gss["fair"].replace(recode_fair)

    table = gss.pivot_table(
        values="fair2", index="year", columns="polviews3", aggfunc="mean"
    )

    for group in groups:
        series = table[group]
        plot_series_lowess(series, color_map[group])

    decorate(
        xlabel="Year",
        ylabel="Fraction saying yes",
        title="Would most people try to be fair?",
    )
\end{lstlisting}

Now we can loop through the three resampled datasets in the data file
and generate a figure for each one.

\begin{lstlisting}[language=Python,style=source]
datafile = "gss_pacs_resampled.hdf"

for key in ["gss0", "gss1", "gss2"]:
    df = pd.read_hdf(datafile, key)
    plt.figure()
    plot_by_polviews(df)
\end{lstlisting}

\begin{center}
\includegraphics[width=4in]{chapters/15_outlook_files/15_outlook_58_0.png}
\end{center}

\begin{center}
\includegraphics[width=4in]{chapters/15_outlook_files/15_outlook_58_1.png}
\end{center}

\begin{center}
\includegraphics[width=4in]{chapters/15_outlook_files/15_outlook_58_2.png}
\end{center}

Features that are the same in all three figures are more likely to
reflect things actually happening in the world. Features that differ
substantially between the figures are more likely to be due to random
sampling.

\textbf{Exercise:} As an exercise, you can run the same analysis with
one of the other variables related to outlook:
\passthrough{\lstinline!happy!}, \passthrough{\lstinline!trust!},
\passthrough{\lstinline!helpful!}, and maybe
\passthrough{\lstinline!fear!} and \passthrough{\lstinline!hapmar!}.

For these variables, you will have to read the codebook to see the
responses and how they are encoded, then think about which responses to
report. In the notebook for this chapter, there are some suggestions to
get you started.

\section{Summary}\label{summary}

The case study in this chapter and the previous one demonstrates a
process for exploring a dataset and finding relationships among the
variables.

In the previous chapter, we started with a single variable,
\passthrough{\lstinline!polviews!}, and visualized its distribution at
the beginning and end of the observation interval. Then we used
\passthrough{\lstinline!groupby!} to see how the mean and standard
deviation changed over time. Looking more closely, we used cross
tabulation to see how the fraction of people in each group changed over
time.

In this chapter, we added a second variable,
\passthrough{\lstinline!fair!}, which is one of several questions in the
GSS related to respondents' beliefs about other people. We used
\passthrough{\lstinline!groupby!} again to see how the responses have
changed over time. Then we used a pivot table to show how the responses
within each political group have changed over time. Finally, we used
multiple resamplings of the original dataset to check whether the
patterns we identified might be due to random sampling rather than real
changes in the world.

The tools we used in this case study are versatile -- they are useful
for exploring other variables in the GSS dataset, and other datasets as
well. And the process we followed is one I recommend whenever you are
exploring a new dataset.

