\section{Introduction}

Data science is the use of data to answer questions and guide decision
making. For example, a topic of current debate is whether we should
raise the minimum wage in the United States. Some economists think that
raising the minimum wage would lift families out of poverty; others
think it would cause more unemployment. But economic theory can only
take us so far. At some point, we need data.

A successful data science project requires three elements:

\begin{itemize}
\item
  A question: for example, what is the relationship between the minimum
  wage and unemployment?
\item
  Data: to answer this question, the best data would result from a
  well designed experiment. But if we can't get ideal data, we have to
  work with what we can get.
\item
  Methods: with the right data, simple methods are often good enough to find
  answers and present them clearly. But sometimes we need more
  specialized tools.
\end{itemize}

In an ideal world, we would pose a question, find data, and choose the
appropriate methods, in that order. More often, the process is
iterative. We might start with one question, get stuck, and pivot to a
different question. Or we might explore a new dataset and discover the
questions it can answer. Or we might start with a tool and look for
problems it can solve. Most data science projects require flexibility
and persistence.

The goal of this book is to give you the tools you need to execute a
data science project from beginning to end, including these steps:

\begin{itemize}
\item
  Choosing questions, data, and methods that go together
\item
  Finding data or collecting it yourself
\item
  Cleaning and validating data
\item
  Exploring datasets and visualizing distributions and relationships
  between variables
\item
  Modeling data and generating predictions
\item
  Designing data visualizations that tell a compelling story
\item
  Communicating results effectively
\end{itemize}

We'll start with basic Python programs and work our way up.
My goal is to present a small,
powerful subset of Python that allows you to do real work in data
science as quickly as possible.

I won't assume you know anything about programming,
statistics, or data science. When I use a term, I'll define it
immediately, and when I use a programming feature, I'll explain it.

For each chapter in this book, there is a Jupyter notebook you can access from \url{https://allendowney.github.io/ElementsOfDataScience}. Jupyter is a software
development tool you can run in a web browser, so you don't have to
install any software. A Jupyter notebook is a document that contains
text, Python code, and results. So you can read it like a book, but you
can also modify the code, run it, develop new programs, and test them.

The notebooks contain exercises where you can practice what you learn.
Most of the exercises are meant to be quick, but a few are more
substantial.

\section{Book Overview}

This book is organized in five parts.

Part I is an introduction to Python with emphasis on concepts and tools for working with data. It introduces Python data structures like lists and dictionaries, NumPy arrays, and Pandas DataFrames.

Part II is about exploratory data analysis, starting with the ways we represent and summarize the distribution of a variable, moving on to relationships between variables, and ending with linear and logistic regression.

Part III is about statistical inference, which is using data to infer the properties of a population. It introduces randomization methods, especially bootstrap resampling, as a tool for estimating a quantity, describing the precision of the estimate, and testing hypotheses.

Part IV is the first of two case studies, an exploration of data from the General Social Survey. In introduces tools for describing changes over time and differences between groups, including cross tabulations and pivot tables.

Part V is the second case study, which introduces classification algorithms and the metrics we use to evaluate their performance. It presents a particularly challenging topic in the criminal justice system, the use of algorithms to predict who is most likely to commit future crimes.

Here are more detailed descriptions of the chapters:

\begin{description}

\item Chapter 1 introduces variables, values, and numerical computation.

\item Chapter 2 shows how to represent times,
dates, and locations in Python, and uses the GeoPandas library to plot
points on a map.

\item Chapter 3 presents lists and NumPy
arrays. It discusses absolute, relative, and percent errors, and ways to
summarize them.

\item Chapter 4 presents the \texttt{for} loop
and the \texttt{if} statement; then it uses them to speed-read \emph{War
and Peace} and count the words.

\item Chapter 5 presents one of the most powerful
features of Python, dictionaries, and uses them to count the unique
words in a text and their frequencies.

\item Chapter 6 introduces a plotting library,
Matplotlib, and uses it to generate a few common data visualizations and
one less common one, a Zipf plot.

\item Chapter 7 presents Pandas DataFrames, which are used
to represent tables of data. As an example, it uses data from the
National Survey of Family Growth to find the average weight of babies in
the U.S.

\item Chapter 8 explains what a distribution is and presents 3 ways to represent one: a PMF, CDF, or PDF. It also shows
how to compare a distribution to another distribution or a mathematical
model.

\item Chapter 9 explores relationships between
variables using scatter plots, violin plots, and box plots. It
quantifies the strength of a relationship using correlation,
and uses simple regression to estimate the slope of a line.

\item Chapter 10 presents multiple regression and uses
it to explore the relationship between age, education, and income. It
uses visualization to interpret multivariate models. It also presents
binary variables and logistic regression.

\item Chapter 11 presents computational methods we can
use to quantify variation due to random sampling, which is one of
several sources of error in statistical estimation.

\item Chapter 12 introduces bootstrapping, a kind of resampling that is
well suited to computational statistics.

\item Chapter 13 uses a computational approach to explain hypothesis testing, which is the bugbear of classical statistics.

\item Chapter 14 is the beginning of a case study that uses data from the General Social Survey. It uses survey responses to explore political polarization in the United States over the last 50 years.

\item Chapter 15 is the second part of the case study; it explores the relationship between political alignment (conservative, moderate, or liberal) and other attitudes and beliefs.

\item Chapter 16 introduces the second case study, related to an algorithm used in the criminal justice system to predict crime. It replicates analysis reported in 2016 to evaluate the performance of the algorithm and its fairness between racial groups.

\item Chapter 17 extends the analysis in the previous chapter to address fairness between men and women, and explores different definitions of fairness and why they are hard to achieve.

\end{description}
