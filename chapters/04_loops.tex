\hypertarget{loops-and-files}{%
\chapter{Loops and files}\label{loops-and-files}}

This chapter presents loops, which are used to express repeated
computation, and files, which are used to store data. As an example, we
will download the famous book \emph{War and Peace} from Project
Gutenberg and write a loop that reads the book and counts the words.

In the next chapter, we'll extend this example to count the number of
unique words and the number of times each word appears. This example
presents some computational tools you will need; it is also an
introduction to working with textual data.

\hypertarget{loops}{%
\section{Loops}\label{loops}}

One of the most important elements of computation is repetition, and the
most common way to express repetition is a \passthrough{\lstinline!for!}
loop. As a simple example, suppose we want to display the elements of a
tuple. Here's a tuple of three integers:

\begin{lstlisting}[language=Python]
t = 1, 2, 3
\end{lstlisting}

And here's a \passthrough{\lstinline!for!} loop that prints the
elements.

\begin{lstlisting}[language=Python]
for x in t:
    print(x)
\end{lstlisting}

\begin{lstlisting}[]
1
2
3
\end{lstlisting}

The first line of the loop is a \textbf{header} that specifies the
tuple, \passthrough{\lstinline!t!}, and a variable name
\passthrough{\lstinline!x!}. The tuple already exists, but
\passthrough{\lstinline!x!} does not; the loop will create it. The
header has to end with a colon, \passthrough{\lstinline!:!}.

Inside the loop is a \passthrough{\lstinline!print!} statement, which
displays the value of \passthrough{\lstinline!x!}.

So here's what happens:

\begin{enumerate}
\def\labelenumi{\arabic{enumi}.}
\item
  When the loop starts, it gets the first element of
  \passthrough{\lstinline!t!}, which is \passthrough{\lstinline!1!}, and
  assigns it to \passthrough{\lstinline!x!}. It executes the
  \passthrough{\lstinline!print!} statement, which displays the value
  \passthrough{\lstinline!1!}.
\item
  Then it gets the second element of \passthrough{\lstinline!t!}, which
  is \passthrough{\lstinline!2!}, and displays it.
\item
  Then it gets the third element of \passthrough{\lstinline!t!}, which
  is \passthrough{\lstinline!3!}, and displays it.
\end{enumerate}

After printing the last element of the tuple, the loop ends.

We can also loop through the letters in a string:

\begin{lstlisting}[language=Python]
word = 'Data'

for letter in word:
    print(letter)
\end{lstlisting}

\begin{lstlisting}[]
D
a
t
a
\end{lstlisting}

When the loop begins, \passthrough{\lstinline!word!} already exists, but
\passthrough{\lstinline!letter!} does not. Again, the loop creates
\passthrough{\lstinline!letter!} and assign values to it.

The variable created by the loop is called the ``loop variable''. You
can give it any name you like; in this example, I chose
\passthrough{\lstinline!letter!} to remind me what kind of value it
contains.

After the loop ends, the loop variable contains the last value.

\begin{lstlisting}[language=Python]
letter
\end{lstlisting}

\begin{lstlisting}[]
'a'
\end{lstlisting}

\textbf{Exercise:} Create a list, called
\passthrough{\lstinline!sequence!} with four elements of any type. Write
a \passthrough{\lstinline!for!} loop that prints the elements. Call the
loop variable \passthrough{\lstinline!element!}.

You might wonder why I didn't call the list
\passthrough{\lstinline!list!}. I avoided it because Python has a
function named \passthrough{\lstinline!list!} that makes new lists. For
example, if you have a string, you can make a list of letters, like
this:

\begin{lstlisting}[language=Python]
list('string')
\end{lstlisting}

\begin{lstlisting}[]
['s', 't', 'r', 'i', 'n', 'g']
\end{lstlisting}

If you create a variable named \passthrough{\lstinline!list!}, you can't
use the function any more.

\hypertarget{looping-and-counting}{%
\section{Looping and counting}\label{looping-and-counting}}

\emph{War and Peace} is a famously long book; let's see how long it is.
To count the words in a book, we need to

\begin{enumerate}
\def\labelenumi{\arabic{enumi}.}
\item
  Loop through a book and
\item
  Count.
\end{enumerate}

We'll start with counting. We've already seen that you can create a
variable and give it a value, like this:

\begin{lstlisting}[language=Python]
count = 0
count
\end{lstlisting}

\begin{lstlisting}[]
0
\end{lstlisting}

If you assign a different value to the same variable, the new value
replaces the old one.

\begin{lstlisting}[language=Python]
count = 1
count
\end{lstlisting}

\begin{lstlisting}[]
1
\end{lstlisting}

You can increase the value of a variable by reading the old value,
adding \passthrough{\lstinline!1!}, and assigning the result back to the
original variable.

\begin{lstlisting}[language=Python]
count = count + 1
count
\end{lstlisting}

\begin{lstlisting}[]
2
\end{lstlisting}

Increasing the value of a variable is called an \textbf{increment};
decreasing the value is called a \textbf{decrement}. These operations
are so common that there are special operators for them.

\begin{lstlisting}[language=Python]
count += 1
count
\end{lstlisting}

\begin{lstlisting}[]
3
\end{lstlisting}

In this example, the \passthrough{\lstinline!+=!} operator reads the
value of \passthrough{\lstinline!count!}, adds
\passthrough{\lstinline!1!}, and assigns the result back to
\passthrough{\lstinline!count!}. Python also provides
\passthrough{\lstinline!-=!} and other update operators like
\passthrough{\lstinline!*=!} and \passthrough{\lstinline!/=!}.

\textbf{Exercise:} The following is a number trick from
\href{https://www.learn-with-math-games.com/math-number-tricks.html}{Learn
With Math Games}

"Finding Someone's Age

\begin{itemize}
\item
  Ask the person to multiply the first number of their age by 5.
\item
  Tell them to add 3.
\item
  Now tell them to double this figure.
\item
  Finally, have the person add the second number of their age to the
  figure and have them tell you the answer.
\item
  Deduct 6 and you will have their age."
\end{itemize}

Test this algorithm using your age. Use a single variable and update it
using \passthrough{\lstinline!+=!} and other update operators.

\hypertarget{files}{%
\section{Files}\label{files}}

Now that we know how to count, let's see how we can read words from a
file. We can download \emph{War and Peace} from
\href{https://www.gutenberg.org}{Project Gutenberg}, which is a
repository of free books.

In order to read the contents of the file, you have to ``open'' it,
which you can do with the \passthrough{\lstinline!open!} function.

\begin{lstlisting}[language=Python]
fp = open('2600-0.txt')
fp
\end{lstlisting}

\begin{lstlisting}[]
<_io.TextIOWrapper name='2600-0.txt' mode='r' encoding='UTF-8'>
\end{lstlisting}

The result is a \passthrough{\lstinline!TextIOWrapper!}, which is a type
of \textbf{file pointer}. The file pointer contains the name of the
file, the mode (which is \passthrough{\lstinline!r!} for ``reading'')
and the encoding (which is \passthrough{\lstinline!UTF!} for ``Unicode
Transformation Format''). A file pointer is like a bookmark; it keeps
track of which parts of the file you have read.

If you use a file pointer in a \passthrough{\lstinline!for!} loop, it
loops through the lines in the file. So we can count the number of lines
like this:

\begin{lstlisting}[language=Python]
fp = open('2600-0.txt')
count = 0
for line in fp:
    count += 1
\end{lstlisting}

And then display the result.

\begin{lstlisting}[language=Python]
count
\end{lstlisting}

\begin{lstlisting}[]
66054
\end{lstlisting}

There are about 66,000 lines in this file.

\hypertarget{if-statements}{%
\section{if statements}\label{if-statements}}

We've already see comparison operators, like \passthrough{\lstinline!>!}
and \passthrough{\lstinline!<!}, which compare values and produce a
Boolean result, \passthrough{\lstinline!True!} or
\passthrough{\lstinline!False!}. For example, we can compare the final
value of \passthrough{\lstinline!count!} to a number:

\begin{lstlisting}[language=Python]
count > 60000
\end{lstlisting}

\begin{lstlisting}[]
True
\end{lstlisting}

We can use a comparison operator in an \passthrough{\lstinline!if!}
statement to check for a condition and take action accordingly.

\begin{lstlisting}[language=Python]
if count > 60000:
    print('Long book!')
\end{lstlisting}

\begin{lstlisting}[]
Long book!
\end{lstlisting}

The first line of the \passthrough{\lstinline!if!} statement specifies
the condition we're checking for. Like the header of a
\passthrough{\lstinline!for!} statement, the first line of an
\passthrough{\lstinline!if!} statement has to end with a colon.

If the condition is true, the indented statement runs; otherwise, it
doesn't. In the previous example, the condition is true, so the
\passthrough{\lstinline!print!} statement runs. In the following
example, the condition is false, so the \passthrough{\lstinline!print!}
statement doesn't run.

\begin{lstlisting}[language=Python]
if count < 1000:
    print('Novella!')
\end{lstlisting}

We can put a \passthrough{\lstinline!print!} statement inside a
\passthrough{\lstinline!for!} loop. In this example, we only print a
line from the book when \passthrough{\lstinline!count!} is
\passthrough{\lstinline!1!}. The other lines are read, but not
displayed.

\begin{lstlisting}[language=Python]
fp = open('2600-0.txt')
count = 0
for line in fp:
    if count == 1:
        print(line)
    count += 1
\end{lstlisting}

\begin{lstlisting}[]
The Project Gutenberg EBook of War and Peace, by Leo Tolstoy
\end{lstlisting}

Notice the indentation in this example:

\begin{itemize}
\item
  Statements inside the \passthrough{\lstinline!for!} loop are indented.
\item
  The statement inside the \passthrough{\lstinline!if!} statement is
  indented.
\item
  The statement \passthrough{\lstinline!count += 1!} is
  \textbf{outdented} from the previous line, so it ends the
  \passthrough{\lstinline!if!} statement. But it is still inside the
  \passthrough{\lstinline!for!} loop.
\end{itemize}

It is legal in Python to use spaces or tabs for indentation, but the
most common convention is to use four spaces, never tabs. That's what
I'll do in my code and I strongly suggest you follow the convention.

\hypertarget{break-statement}{%
\section{break statement}\label{break-statement}}

If we display the final value of \passthrough{\lstinline!count!}, we see
that the loop reads the entire file, but only prints one line:

\begin{lstlisting}[language=Python]
count
\end{lstlisting}

\begin{lstlisting}[]
66054
\end{lstlisting}

We can avoid reading the whole file by using a
\passthrough{\lstinline!break!} statement, like this:

\begin{lstlisting}[language=Python]
fp = open('2600-0.txt')
count = 0
for line in fp:
    if count == 1:
        print(line)
        break
    count += 1
\end{lstlisting}

\begin{lstlisting}[]
The Project Gutenberg EBook of War and Peace, by Leo Tolstoy
\end{lstlisting}

The \passthrough{\lstinline!break!} statement ends the loop immediately,
skipping the rest of the file. We can confirm that by checking the last
value of \passthrough{\lstinline!count!}:

\begin{lstlisting}[language=Python]
count
\end{lstlisting}

\begin{lstlisting}[]
1
\end{lstlisting}

\textbf{Exercise:} Write a loop that prints the first 5 lines of the
file and then breaks out of the loop.

\hypertarget{whitespace}{%
\section{Whitespace}\label{whitespace}}

If we run the loop again and display the final value of
\passthrough{\lstinline!line!}, we see the special sequence
\passthrough{\lstinline!\\n!} at the end.

\begin{lstlisting}[language=Python]
fp = open('2600-0.txt')
count = 0
for line in fp:
    if count == 1:
        break
    count += 1

line
\end{lstlisting}

\begin{lstlisting}[]
'The Project Gutenberg EBook of War and Peace, by Leo Tolstoy\n'
\end{lstlisting}

This sequence represents a single character, called a \textbf{newline},
that puts vertical space between lines. If we use a
\passthrough{\lstinline!print!} statement to display
\passthrough{\lstinline!line!}, we don't see the special sequence, but
we do see extra space after the line.

\begin{lstlisting}[language=Python]
print(line)
\end{lstlisting}

\begin{lstlisting}[]
The Project Gutenberg EBook of War and Peace, by Leo Tolstoy
\end{lstlisting}

In other strings, you might see the sequence
\passthrough{\lstinline!\\t!}, which represents a ``tab'' character.
When you print a tab character, it adds enough space to make the next
character appear in a column that is a multiple of 8.

\begin{lstlisting}[language=Python]
print('01234567' * 6)
print('a\tbc\tdef\tghij\tklmno\tpqrstu')
\end{lstlisting}

\begin{lstlisting}[]
012345670123456701234567012345670123456701234567
a   bc  def ghij    klmno   pqrstu
\end{lstlisting}

Newline characters, tabs, and spaces are called \textbf{whitespace}
because when they are printed they leave white space on the page
(assuming that the background color is white).

\hypertarget{counting-words}{%
\section{Counting words}\label{counting-words}}

So far we've managed to count the lines in a file, but each line
contains several words. To split a line into words, we can use a
function called \passthrough{\lstinline!split!} that returns a list of
words. To be more precise, \passthrough{\lstinline!split!} doesn't
actually know what a word is; it just splits the line wherever there's a
space or other whitespace character.

\begin{lstlisting}[language=Python]
line.split()
\end{lstlisting}

\begin{lstlisting}[]
['The',
 'Project',
 'Gutenberg',
 'EBook',
 'of',
 'War',
 'and',
 'Peace,',
 'by',
 'Leo',
 'Tolstoy']
\end{lstlisting}

Notice that the syntax for \passthrough{\lstinline!split!} is different
from other functions we have seen. Normally when we call a function, we
name the function and provide values in parentheses. So you might have
expected to write \passthrough{\lstinline!split(line)!}. Sadly, that
doesn't work.

The problem is that the \passthrough{\lstinline!split!} function belongs
to the string \passthrough{\lstinline!line!}; in a sense, the function
is attached to the string, so we can only refer to it using the string
and the dot operator (the period between \passthrough{\lstinline!line!}
and \passthrough{\lstinline!split!}). For historical reasons, functions
like this are called \textbf{methods}.

Now that we can split a line into a list of words, we can use
\passthrough{\lstinline!len!} to get the number of words in each list,
and increment \passthrough{\lstinline!count!} accordingly.

\begin{lstlisting}[language=Python]
fp = open('2600-0.txt')
count = 0
for line in fp:
    count += len(line.split())
\end{lstlisting}

\begin{lstlisting}[language=Python]
count
\end{lstlisting}

\begin{lstlisting}[]
566317
\end{lstlisting}

By this count, there are more than half a million words in \emph{War and
Peace}.

Actually, there aren't quite that many, because the file we got from
Project Gutenberg has some introductory text and a table of contents
before the text. And it has some license information at the end. To skip
this ``front matter'', we can use one loop to read lines until we get to
\passthrough{\lstinline!CHAPTER I!}, and then a second loop to count the
words in the remaining lines.

The file pointer, \passthrough{\lstinline!fp!}, keeps track of where it
is in the file, so the second loop picks up where the first loop leaves
off. In the second loop, we check for the end of the book and stop, so
we ignore the ``end matter'' at the end of the file.

\begin{lstlisting}[language=Python]
first_line = "CHAPTER I\n"
last_line = "End of the Project Gutenberg EBook of War and Peace, by Leo Tolstoy\n"

fp = open('2600-0.txt')
for line in fp:
    if line == first_line:
        break

count = 0
for line in fp:
    if line == last_line:
        print(line)
        break
    count += len(line.split())
\end{lstlisting}

\begin{lstlisting}[]
End of the Project Gutenberg EBook of War and Peace, by Leo Tolstoy
\end{lstlisting}

\begin{lstlisting}[language=Python]
count
\end{lstlisting}

\begin{lstlisting}[]
562482
\end{lstlisting}

Two things to notice about this program:

\begin{itemize}
\item
  When we compare two values to see if they are equal, we use the
  \passthrough{\lstinline!==!} operator, not to be confused with
  \passthrough{\lstinline!=!}, which is the assignment operator.
\item
  The string we compare \passthrough{\lstinline!line!} to has a newline
  at the end. If we leave that out, it doesn't work.
\end{itemize}

\textbf{Exercise:}

\begin{enumerate}
\def\labelenumi{\arabic{enumi}.}
\item
  In the previous program, replace \passthrough{\lstinline!==!} with
  \passthrough{\lstinline!=!} and see what happens. This is a common
  error, so it is good to see what the error message looks like.
\item
  Correct the previous error, then remove the newline character after
  \passthrough{\lstinline!CHAPTER I!}, and see what happens.
\end{enumerate}

The first error is a ``syntax error'', which means that the program
violates the rules of Python. If your program has a syntax error, the
Python interpreter prints an error message, and the program never runs.

The second error is a ``logic error'', which means that there is
something wrong with the logic of the program. The syntax is legal, and
the program runs, but it doesn't do what we wanted. Logic errors can be
hard to find because we don't get any error messages.

If you have a logic error, here are two strategies for debugging:

\begin{enumerate}
\def\labelenumi{\arabic{enumi}.}
\item
  Add print statements so the program displays additional information
  while it runs.
\item
  Simplify the program until it does what you expect, and then gradually
  add more code, testing as you go.
\end{enumerate}

