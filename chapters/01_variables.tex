\hypertarget{variables-and-values}{%
\chapter{Variables and Values}\label{variables-and-values}}

Data science is the use of data to answers questions and guide decision
making. For example, a topic of current debate is whether we should
raise the minimum wage in the United States. Some economists think that
raising the minimum wage would raise families out of poverty; others
think it would cause more unemployment. But economic theory can only
take us so far. At some point, we need data.

A successful data science project requires three elements:

\begin{itemize}
\item
  A question: For example, what is the relationship between the minimum
  wage and unemployment?
\item
  Data: To answer this question, the best data would be results from a
  well designed experiment. But if we can't get ideal data, we have to
  work with what we can get.
\item
  Methods: With the right data, simple methods are often enough to find
  answers and present them clearly. But sometimes we need more
  specialized tools.
\end{itemize}

In an ideal world, we would pose a question, find data, and choose the
appropriate methods, in that order. More often, the process is
iterative. We might start with one question, get stuck, and pivot to a
different question. Or we might explore a new dataset and discover the
questions it can answer. Or we might start with a tool and look for
problems it can solve. Most data science projects require flexibility
and persistence.

The goal of this book is to give you the tools you need to execute a
data science project from beginning to end, including these steps:

\begin{itemize}
\item
  Choosing questions, data, and methods that go together.
\item
  Finding data or collecting it yourself.
\item
  Cleaning and validating data.
\item
  Exploring datasets, visualizing distributions and relationships
  between variables.
\item
  Modeling data and generating predictions.
\item
  Designing data visualizations that tell a compelling story.
\item
  Communicating results effectively.
\end{itemize}

We'll start with basic programming concepts and work our way toward data
science tools.

I won't assume that you already know about programming, statistics, or
data science. When I use a term, I try to define it immediately, and
when I use a programming feature, I try to explain it clearly.

This book is in the form of Jupyter notebooks. Jupyter is a software
development tool you can run in a web browser, so you don't have to
install any software. A Jupyter notebook is a document that contains
text, Python code, and results. So you can read it like a book, but you
can also modify the code, run it, develop new programs, and test them.

The notebooks contain exercises where you can practice what you learn. I
encourage you to do the exercises as you go along.

The topics in this chapter are:

\begin{itemize}
\item
  Using Jupyter to write and run Python code.
\item
  Basic programming features in Python: variables and values.
\item
  Translating formulas from math notation to Python.
\end{itemize}

Along the way, we'll review a couple of math topics I assume you have
seen before, logarithms and algebra.

\hypertarget{numbers}{%
\section{Numbers}\label{numbers}}

Python provides tools for working with many kinds of data, including
numbers, words, dates, times, and locations (latitude and longitude).

Let's start with numbers. Python can work with several types of numbers,
but the two most common are:

\begin{itemize}
\item
  \passthrough{\lstinline!int!}, which represents integer values like
  \passthrough{\lstinline!3!}, and
\item
  \passthrough{\lstinline!float!}, which represents numbers that have a
  fraction part, like \passthrough{\lstinline!3.14159!}.
\end{itemize}

Most often, we use \passthrough{\lstinline!int!} to represent counts and
\passthrough{\lstinline!float!} to represent measurements. Here's an
example of an \passthrough{\lstinline!int!} and a
\passthrough{\lstinline!float!}:

\begin{lstlisting}[]
3
(@\dashfill@)
@@@3@@@
\end{lstlisting}

\begin{lstlisting}[]
3.14159
(@\dashfill@)
@@@3.14159@@@
\end{lstlisting}

\passthrough{\lstinline!float!} is short for ``floating-point'', which
is the name for the way these numbers are stored.

\hypertarget{arithmetic}{%
\section{Arithmetic}\label{arithmetic}}

Python provides operators that perform arithmetic. The operators that
perform addition and subtraction are \passthrough{\lstinline!+!} and
\passthrough{\lstinline!-!}:

\begin{lstlisting}[]
3 + 2 - 1
(@\dashfill@)
@@@4@@@
\end{lstlisting}

The operators that perform multiplication and division are
\passthrough{\lstinline!*!} and \passthrough{\lstinline!/!}:

\begin{lstlisting}[]
2 * 3
(@\dashfill@)
@@@6@@@
\end{lstlisting}

\begin{lstlisting}[]
2 / 3
(@\dashfill@)
@@@0.6666666666666666@@@
\end{lstlisting}

And the operator for exponentiation is \passthrough{\lstinline!**!}:

\begin{lstlisting}[]
2**3
(@\dashfill@)
@@@8@@@
\end{lstlisting}

Unlike math notation, Python does not allow ``implicit multiplication''.
For example, in math notation, if you write \(3 (2 + 1)\), that's
understood to be the same as \(3 \times (2+ 1)\). Python does not allow
that notation. If you want to multiply, you have to use the
\passthrough{\lstinline!*!} operator.

The arithmetic operators follow the rules of precedence you might have
learned as ``PEMDAS'':

\begin{itemize}

\item
  Parentheses before
\item
  Exponentiation before
\item
  Multiplication and division before
\item
  Addition and subtraction.
\end{itemize}

So in this expression:

\begin{lstlisting}[]
1 + 2 * 3
(@\dashfill@)
@@@7@@@
\end{lstlisting}

The multiplication happens first. If that's not what you want, you can
use parentheses to make the order of operations explicit:

\begin{lstlisting}[]
(1 + 2) * 3
(@\dashfill@)
@@@9@@@
\end{lstlisting}

\textbf{Exercise:} Write a Python expression that raises
\passthrough{\lstinline!1+2!} to the power
\passthrough{\lstinline!3*4!}. The answer should be
\passthrough{\lstinline!531441!}.

\hypertarget{math-functions}{%
\section{Math Functions}\label{math-functions}}

Python provides functions that compute all the usual mathematical
functions, like \passthrough{\lstinline!sin!} and
\passthrough{\lstinline!cos!}, \passthrough{\lstinline!exp!} and
\passthrough{\lstinline!log!}. However, they are not part of Python
itself; they are in a \textbf{library}, which is a collection of
functions that supplement the Python language.

Actually, there are several libraries that provide math functions; the
one we'll use is called NumPy, which stands for ``Numerical Python'',
and is pronounced ``num pie''. Before you can use a library, you have to
\textbf{import} it. Here's how we import NumPy:

\begin{lstlisting}[]
import numpy as np
\end{lstlisting}

It is conventional to import \passthrough{\lstinline!numpy!} as
\passthrough{\lstinline!np!}, which means we can refer to it by the
short name \passthrough{\lstinline!np!} rather than the longer name
\passthrough{\lstinline!numpy!}. Names like this are case-sensitive,
which means that \passthrough{\lstinline!numpy!} is not the same as
\passthrough{\lstinline!NumPy!}. So even though the name of the library
is NumPy, when we import it we have to call it
\passthrough{\lstinline!numpy!}.

But assuming we import \passthrough{\lstinline!np!} correctly, we can
use it to read the value \passthrough{\lstinline!pi!}, which is an
approximation of the mathematical constant \(\pi\).

\begin{lstlisting}[]
np.pi
(@\dashfill@)
@@@3.141592653589793@@@
\end{lstlisting}

The result is a \passthrough{\lstinline!float!} with 16 digits. As you
might know, we can't represent \(\pi\) with a finite number of digits,
so this result is only approximate.

NumPy provides \passthrough{\lstinline!log!}, which computes the natural
logarithm

\begin{lstlisting}[]
np.log(100)
(@\dashfill@)
@@@4.605170185988092@@@
\end{lstlisting}

NumPy also provides \passthrough{\lstinline!exp!}, which raises the
constant \passthrough{\lstinline!e!} to a power.

\begin{lstlisting}[]
np.exp(1)
(@\dashfill@)
@@@2.718281828459045@@@
\end{lstlisting}

\textbf{Exercise:} Use these functions to confirm the mathematical
identity \(\log(e^x) = x\), which should be true for any value of \(x\).

With floating-point values, this identity should work for values of
\(x\) between -700 and 700. What happens when you try it with larger and
smaller values?

As this example shows, floating-point numbers are finite approximations,
which means they don't always behave like math.

As another example, let's see what happens when you add up
\passthrough{\lstinline!0.1!} three times:

\begin{lstlisting}[]
0.1 + 0.1 + 0.1
(@\dashfill@)
@@@0.30000000000000004@@@
\end{lstlisting}

The result is close to \passthrough{\lstinline!0.3!}, but not exact.
We'll see other examples of floating-point approximation later, and
learn some ways to deal with it.

\hypertarget{variables}{%
\section{Variables}\label{variables}}

A \textbf{variable} is a name that refers to a value. The following
statement assigns the value \passthrough{\lstinline!5!} to a variable
named \passthrough{\lstinline!x!}:

\begin{lstlisting}[]
x = 5
\end{lstlisting}

The variable we just created has the name \passthrough{\lstinline!x!}
and the value \passthrough{\lstinline!5!}.

If we use \passthrough{\lstinline!x!} as part of an arithmetic
operation, it represents the value \passthrough{\lstinline!5!}:

\begin{lstlisting}[]
x + 1
(@\dashfill@)
@@@6@@@
\end{lstlisting}

\begin{lstlisting}[]
x**2
(@\dashfill@)
@@@25@@@
\end{lstlisting}

We can also use \passthrough{\lstinline!x!} with
\passthrough{\lstinline!numpy!} functions:

\begin{lstlisting}[]
np.exp(x)
(@\dashfill@)
@@@148.4131591025766@@@
\end{lstlisting}

Notice that the result from \passthrough{\lstinline!exp!} is a
\passthrough{\lstinline!float!}, even though the value of
\passthrough{\lstinline!x!} is an \passthrough{\lstinline!int!}.

\textbf{Exercise:} If you have not programmed before, one of the things
you have to get used to is that programming languages are picky about
details. Natural languages, like English, and semi-formal languages,
like math notation, are more forgiving.

As an example, in math notation, parentheses and square brackets mean
the same thing, you can write

\(\sin (\omega t)\)

or

\(\sin [\omega t]\)

Either one is fine. And you can leave out the parentheses altogether, as
long as the meaning is clear:

\(\sin \omega t\)

In Python, every character counts. For example, the following are all
different, and only the first one works:

\begin{lstlisting}[style=output]
np.exp(x)
np.Exp(x)
np.exp[x]
np.exp x
\end{lstlisting}

While you are learning, I encourage you to make mistakes on purpose to
see what goes wrong. Read the error messages carefully. Sometimes they
are helpful and tell you exactly what's wrong. Other times they can be
misleading. But if you have seen the message before, you might remember
some likely causes.

\textbf{Exercise:} Search the NumPy documentation to find the function
that computes square roots, and use it to compute a floating-point
approximation of the golden ratio:

\(\phi = \frac{1 + \sqrt{5}}{2}\)

Hint: The result should be close to \passthrough{\lstinline!1.618!}.

\hypertarget{calculating-with-variables}{%
\section{Calculating with Variables}\label{calculating-with-variables}}

Now we'll use variables to solve a problem involving compound interest.
It might not be the most exciting example, but it uses everything we
have done so far, and it reviews exponentiation and logarithms, which we
are going to need.

If we start with a principal sum, \(P\), and earn compounded interest,
the total accumulated value, \(V\), at the end of time \(t\) is:

\(V=P\left(1+{\frac {r}{n}}\right)^{nt}\)

where \(r\) is the annual interest rate and \(n\) is the compounding
frequency.

For example, if you deposit \$2,100 in a bank paying an annual interest
rate of 3.4\% compounded quarterly, we can compute the balance after 7
years by defining these variables:

\begin{lstlisting}[]
P = 2100
r = 0.034
n = 4
t = 7
\end{lstlisting}

And computing the total accumulated value like this.

\begin{lstlisting}[]
P * (1 + r/n)**(n*t)
(@\dashfill@)
@@@2661.6108980682593@@@
\end{lstlisting}

\textbf{Exercise:} Continuing the previous example, suppose you start
with the same principle and the same interest rate, but interest is
compounded twice per year, so \passthrough{\lstinline!n = 2!}. What
would the total value be after 7 years? Hint: we expect the answer to be
a bit less than the previous answer.

\textbf{Exercise:} If interest is compounded continuously, the value
after time \(t\) is given by the formula:

\(V=P~e^{rt}\)

Translate this equation into Python and use it compute the value of the
investment in the previous example with continuous compounding. Hint: we
expect the answer to be a bit more than the previous answers.

\hypertarget{summary}{%
\section{Summary}\label{summary}}

This chapter introduces variables, which are names that refer to values,
and two kinds of values, integers and floating-point numbers.

It presents mathematical operators, like \passthrough{\lstinline!+!} for
addition and \passthrough{\lstinline!*!} for multiplication, and
mathematical functions like \passthrough{\lstinline!log!} for logarithms
and \passthrough{\lstinline!exp!} for raising
\passthrough{\lstinline!e!} to a power.

In the next chapter, we'll see additional data types for representing
letters and words, dates and times, and latitude and longitude.

