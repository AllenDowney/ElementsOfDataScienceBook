\hypertarget{distributions}{%
\chapter{Distributions}\label{distributions}}

In this chapter we'll see three ways to describe a distribution:

\begin{itemize}
\item
  A probability mass function (PMF), which represents a set of values
  and the number of times each one appears in a dataset.
\item
  A cumulative distribution function (CDF), which contains the same
  information as a PMF in a form that makes it easier to visualize, make
  comparisons, and perform some computations.
\item
  A kernel density estimate (KDE), which is like a smooth, continuous
  version of a histogram.
\end{itemize}

As examples, we'll use data from the General Social Survey (GSS) to look
at distributions of age and income, and to explore the relationship
between income and education. But we'll start with one of the most
important ideas in statistics, the distribution.

\hypertarget{distributions-1}{%
\section{Distributions}\label{distributions-1}}

A distribution is a set of values and their corresponding probabilities.
For example, if you roll a six-sided die, there are six possible
outcomes, the numbers \passthrough{\lstinline!1!} through
\passthrough{\lstinline!6!}, and they all have the same probability,
\passthrough{\lstinline!1/6!}. We can represent this distribution of
outcomes with a table, like this:

\begin{longtable}[]{@{}ll@{}}
\midrule()
Value & Probability \\
\midrule()
\endhead
1 & 1/6 \\
2 & 1/6 \\
3 & 1/6 \\
4 & 1/6 \\
5 & 1/6 \\
6 & 1/6 \\
\midrule()
\end{longtable}

More generally, a distribution can have any number of values, the values
can be any type, and the probabilities do not have to be equal. To
represent distributions in Python, we will use a library called
\passthrough{\lstinline!empiricaldist!}, for ``empirical distribution'',
where ``empirical'' means it is based on data rather than a mathematical
formula.

\passthrough{\lstinline!empiricaldist!} provides an object called
\passthrough{\lstinline!Pmf!}, which stands for ``probability mass
function''. A \passthrough{\lstinline!Pmf!} object contains a set of
possible outcomes and their probabilities. For example, here's a
\passthrough{\lstinline!Pmf!} that represents the outcome of rolling a
six-sided die:

\begin{lstlisting}[language=Python,style=source]
from empiricaldist import Pmf

outcomes = [1,2,3,4,5,6]
die = Pmf(1/6, outcomes)
\end{lstlisting}

The first argument is the probability of each outcome; the second
argument is the list of outcomes. We can display the result like this.

\begin{lstlisting}[language=Python,style=source]
die
\end{lstlisting}

\begin{lstlisting}[style=output]
1    0.166667
2    0.166667
3    0.166667
4    0.166667
5    0.166667
6    0.166667
dtype: float64
\end{lstlisting}

A \passthrough{\lstinline!Pmf!} object is a specialized version of a
Pandas \passthrough{\lstinline!Series!}, so it provides all of the
attributes and methods of a \passthrough{\lstinline!Series!}, plus some
additional methods we'll see soon.

\hypertarget{the-general-social-survey}{%
\section{The General Social Survey}\label{the-general-social-survey}}

Now we'll use \passthrough{\lstinline!Pmf!} objects to represent
distributions of values from a new dataset, the General Social Survey
(GSS). The GSS surveys a representative sample of adult residents of the
U.S. and asks questions about demographics, personal history, and
beliefs about social and political issues. It is widely used by
politicians, policy makers, and researchers.

The GSS dataset contains hundreds of columns; using an online tool call
\href{https://gssdataexplorer.norc.org/}{GSS Explorer} I've selected
just a few and created a subset of the data, called an \textbf{extract}.
I have stored the data in an HDF file, which is more compact than the
original fixed-width file, and faster to read. Instructions for
downloading the files are in the notebook for this chapter.

\begin{lstlisting}[language=Python,style=source]
data_file = 'gss_extract_2022.hdf'
\end{lstlisting}

We'll use \passthrough{\lstinline!pd.read\_hdf!} to read the data file
and extract from it \passthrough{\lstinline!gss!}, which is a
\passthrough{\lstinline!DataFrame!}.

\begin{lstlisting}[language=Python,style=source]
import pandas as pd

gss = pd.read_hdf(data_file, 'gss')
gss.shape
\end{lstlisting}

\begin{lstlisting}[style=output]
(72390, 9)
\end{lstlisting}

The result is a \passthrough{\lstinline!DataFrame!} with one row for
each respondent, and one column for each question in the extract. Here
are the first few rows.

\begin{lstlisting}[language=Python,style=source]
gss.head()
\end{lstlisting}

\begin{lstlisting}[style=output]
   year  id   age  educ  degree  sex  gunlaw  grass  realinc
0  1972   1  23.0  16.0     3.0  2.0     1.0    NaN  18951.0
1  1972   2  70.0  10.0     0.0  1.0     1.0    NaN  24366.0
2  1972   3  48.0  12.0     1.0  2.0     1.0    NaN  24366.0
3  1972   4  27.0  17.0     3.0  2.0     1.0    NaN  30458.0
4  1972   5  61.0  12.0     1.0  2.0     1.0    NaN  50763.0
\end{lstlisting}

I'll explain these columns as we go along, but if you want more
information, you can read the online documentation at
\url{https://gssdataexplorer.norc.org/variables/vfilter}. In the GSS
documentation, you'll see that they use the term ``variable'' for a
column that contains answers to survey questions.

\hypertarget{distribution-of-education}{%
\section{Distribution of Education}\label{distribution-of-education}}

To get started with this dataset, let's look at the distribution of
\passthrough{\lstinline!educ!}, which records the number of years of
education for each respondent. First we'll select a column from the
\passthrough{\lstinline!DataFrame!} and use
\passthrough{\lstinline!value\_counts!} to see what values are in it.

\begin{lstlisting}[language=Python,style=source]
gss['educ'].value_counts().sort_index()
\end{lstlisting}

\begin{lstlisting}[style=output]
educ
0.0       177
1.0        49
2.0       158
3.0       268
4.0       326
5.0       410
6.0       866
7.0       896
8.0      2786
9.0      2172
10.0     3010
11.0     3942
12.0    21401
13.0     5905
14.0     8208
15.0     3307
16.0     9994
17.0     2392
18.0     2945
19.0     1112
20.0     1803
Name: count, dtype: int64
\end{lstlisting}

The result from \passthrough{\lstinline!value\_counts!} is a set of
possible values and the number of times each one appears, so it is a
kind of distribution. The values \passthrough{\lstinline!98!} and
\passthrough{\lstinline!99!} are special codes for ``Don't know'' and
``No answer''. We'll use \passthrough{\lstinline!replace!} to replace
these codes with \passthrough{\lstinline!NaN!}.

\begin{lstlisting}[language=Python,style=source]
import numpy as np

educ = gss['educ'].replace([98, 99], np.nan)
\end{lstlisting}

\passthrough{\lstinline!replace!} creates a new
\passthrough{\lstinline!Series!} -- it does not modify the column in the
\passthrough{\lstinline!DataFrame!}. Here's a histogram of the values in
this \passthrough{\lstinline!Series!}.

\begin{lstlisting}[language=Python,style=source]
import matplotlib.pyplot as plt

educ.hist(grid=False)
plt.xlabel('Years of education')
plt.ylabel('Number of respondents')
plt.title('Histogram of education level');
\end{lstlisting}

\begin{center}
\includegraphics[width=4in]{chapters/08_distributions_files/08_distributions_28_0.png}
\end{center}

Based on the histogram, we can see the general shape of the distribution
and the central tendency -- it looks like the peak is near 12 years of
education. But a histogram is not the best way to visualize this
distribution because it obscures some important details. An alternative
is to use a \passthrough{\lstinline!Pmf!}. The function
\passthrough{\lstinline!Pmf.from\_seq!} takes any kind of sequence --
like a list, tuple, or Pandas \passthrough{\lstinline!Series!} -- and
computes the distribution of the values in the sequence.

\begin{lstlisting}[language=Python,style=source]
pmf_educ = Pmf.from_seq(educ, normalize=False)
type(pmf_educ)
\end{lstlisting}

\begin{lstlisting}[style=output]
empiricaldist.empiricaldist.Pmf
\end{lstlisting}

The keyword argument \passthrough{\lstinline!normalize=False!} indicates
that we don't want to normalize this PMF. I'll explain what that means
soon. Here are the first few rows.

\begin{lstlisting}[language=Python,style=source]
pmf_educ.head()
\end{lstlisting}

\begin{lstlisting}[style=output]
educ
0.0    177
1.0     49
2.0    158
Name: count, dtype: int64
\end{lstlisting}

In this dataset, there are \passthrough{\lstinline!165!} respondents who
report that they have had no formal education, and
\passthrough{\lstinline!47!} who have only one year. Here the last few
rows.

\begin{lstlisting}[language=Python,style=source]
pmf_educ.tail()
\end{lstlisting}

\begin{lstlisting}[style=output]
educ
18.0    2945
19.0    1112
20.0    1803
Name: count, dtype: int64
\end{lstlisting}

There are \passthrough{\lstinline!1439!} respondents who report that
they have 20 or more years of formal education, which probably means
they attended college and graduate school.

You can use the bracket operator to look up a value in a
\passthrough{\lstinline!Pmf!} and get the corresponding count:

\begin{lstlisting}[language=Python,style=source]
pmf_educ[20]
\end{lstlisting}

\begin{lstlisting}[style=output]
1803
\end{lstlisting}

Usually when we make a PMF, we want to know the \emph{fraction} of
respondents with each value, rather than the counts. We can do that by
setting \passthrough{\lstinline!normalize=True!}. Then we get a
\textbf{normalized} PMF, that is, a PMF where the values in the second
column add up to 1.

\begin{lstlisting}[language=Python,style=source]
pmf_educ_norm = Pmf.from_seq(educ, normalize=True)
pmf_educ_norm.head()
\end{lstlisting}

\begin{lstlisting}[style=output]
educ
0.0    0.002454
1.0    0.000679
2.0    0.002191
Name: proportion, dtype: float64
\end{lstlisting}

Now if we use the bracket operator to look up a value, the result is a
fraction rather than a count. For example, the fraction of people with
\passthrough{\lstinline!12!} years of education is about \(30\%\):

\begin{lstlisting}[language=Python,style=source]
pmf_educ_norm[12]
\end{lstlisting}

\begin{lstlisting}[style=output]
0.2967127428008929
\end{lstlisting}

\passthrough{\lstinline!Pmf!} provides a \passthrough{\lstinline!bar!}
method that plots the values and their probabilities as a bar chart.

\begin{lstlisting}[language=Python,style=source]
pmf_educ_norm.bar(label='educ')

plt.xlabel('Years of education')
plt.xticks(range(0, 21, 4))
plt.ylabel('PMF')
plt.title('Distribution of years of education')
plt.legend();
\end{lstlisting}

\begin{center}
\includegraphics[width=4in]{chapters/08_distributions_files/08_distributions_42_0.png}
\end{center}

In this figure, we can see that the most common value is
\passthrough{\lstinline!12!} years, but there are also peaks at
\passthrough{\lstinline!14!} and \passthrough{\lstinline!16!}, which
correspond to two and four years of college. For this data, the PMF is
probably a better choice than the histogram. The PMF shows all unique
values, so we can see where the peaks are. Because the histogram puts
values into bins, it obscures these details. In a histogram with the
default number of bins, we couldn't see the peaks at
\passthrough{\lstinline!14!} and \passthrough{\lstinline!16!} years. But
PMFs have limitations, too, as we'll see. First, here's an exercise
where you can practice with PMFs.

\textbf{Exercise:} Let's look at the \passthrough{\lstinline!year!}
column in the \passthrough{\lstinline!DataFrame!}, which represents the
year each respondent was interviewed. Make an unnormalized
\passthrough{\lstinline!Pmf!} for \passthrough{\lstinline!year!} and
display the result. Use the bracket operator to look up the number of
respondents interviewed in 2022?

\hypertarget{cumulative-distribution-functions}{%
\section{Cumulative Distribution
Functions}\label{cumulative-distribution-functions}}

Now we'll see another way to represent a distribution, the cumulative
distribution function (CDF). \passthrough{\lstinline!empiricaldist!}
provides a \passthrough{\lstinline!Cdf!} object that represents a CDF.
We can import it like this:

\begin{lstlisting}[language=Python,style=source]
from empiricaldist import Cdf
\end{lstlisting}

As an example, suppose we have a sequence of five values:

\begin{lstlisting}[language=Python,style=source]
values = 1, 2, 2, 3, 5  
\end{lstlisting}

Here's the \passthrough{\lstinline!Pmf!} of these values.

\begin{lstlisting}[language=Python,style=source]
Pmf.from_seq(values)
\end{lstlisting}

\begin{lstlisting}[style=output]
1    0.2
2    0.4
3    0.2
5    0.2
Name: proportion, dtype: float64
\end{lstlisting}

If you draw a random value from \passthrough{\lstinline!values!}, the
\passthrough{\lstinline!Pmf!} tells you the chance of getting
\passthrough{\lstinline!x!}, for any value of
\passthrough{\lstinline!x!}.

\begin{itemize}
\item
  So the probability of the value \passthrough{\lstinline!1!} is
  \passthrough{\lstinline!1/5!},
\item
  The probability of the value \passthrough{\lstinline!2!} is
  \passthrough{\lstinline!2/5!}, and
\item
  The probabilities for \passthrough{\lstinline!3!} and
  \passthrough{\lstinline!5!} are \passthrough{\lstinline!1/5!} each.
\end{itemize}

A CDF is similar to a PMF in the sense that it contains values and their
probabilities -- the difference is that the probabilities in the CDF are
the cumulative sum of the probabilities in the PMF. Here's a
\passthrough{\lstinline!Cdf!} object for the same five values.

\begin{lstlisting}[language=Python,style=source]
Cdf.from_seq(values)
\end{lstlisting}

\begin{lstlisting}[style=output]
1    0.2
2    0.6
3    0.8
5    1.0
Name: count, dtype: float64
\end{lstlisting}

If you draw a random value from \passthrough{\lstinline!values!},
\passthrough{\lstinline!Cdf!} tells you the chance of getting a value
\emph{less than or equal to} \passthrough{\lstinline!x!}, for any given
\passthrough{\lstinline!x!}.

\begin{itemize}
\item
  So the \passthrough{\lstinline!Cdf!} of \passthrough{\lstinline!1!} is
  \passthrough{\lstinline!1/5!} because one of the five values in the
  sequence is less than or equal to 1,
\item
  The \passthrough{\lstinline!Cdf!} of \passthrough{\lstinline!2!} is
  \passthrough{\lstinline!3/5!} because three of the five values are
  less than or equal to \passthrough{\lstinline!2!},
\item
  And the \passthrough{\lstinline!Cdf!} of \passthrough{\lstinline!5!}
  is \passthrough{\lstinline!5/5!} because all of the values are less
  than or equal to \passthrough{\lstinline!5!}.
\end{itemize}

\hypertarget{cdf-of-age}{%
\section{CDF of Age}\label{cdf-of-age}}

Now let's look at a more substantial \passthrough{\lstinline!Cdf!}, the
distribution of ages for respondents in the General Social Survey. The
column we'll use is \passthrough{\lstinline!age!}.

According to the codebook, the range of the values is from
\passthrough{\lstinline!18!} to \passthrough{\lstinline!89!}, where
\passthrough{\lstinline!89!} means ``89 or older''. The special codes
\passthrough{\lstinline!98!} and \passthrough{\lstinline!99!} mean
``Don't know'' and ``Didn't answer''. We can use
\passthrough{\lstinline!replace!} to replace the special codes with
\passthrough{\lstinline!NaN!}.

\begin{lstlisting}[language=Python,style=source]
age = gss['age'].replace([98, 99], np.nan)
\end{lstlisting}

We can compute the \passthrough{\lstinline!Cdf!} of these values like
this:

\begin{lstlisting}[language=Python,style=source]
cdf_age = Cdf.from_seq(age)
\end{lstlisting}

\passthrough{\lstinline!Cdf!} provides a method called
\passthrough{\lstinline!plot!} that plots the CDF as a line. Here's what
it looks like.

\begin{lstlisting}[language=Python,style=source]
cdf_age.plot()

plt.xlabel('Age (years)')
plt.ylabel('CDF')
plt.title('Distribution of age');
\end{lstlisting}

\begin{center}
\includegraphics[width=4in]{chapters/08_distributions_files/08_distributions_61_0.png}
\end{center}

The \(x\)-axis is the ages, from 18 to 89. The \(y\)-axis is the
cumulative probabilities, from 0 to 1.

\passthrough{\lstinline!cdf\_age!} can be used as a function, so if you
give it an age, it returns the corresponding probability (in a NumPy
array).

\begin{lstlisting}[language=Python,style=source]
q = 51
p = cdf_age(q)
p
\end{lstlisting}

\begin{lstlisting}[style=output]
array(0.62121445)
\end{lstlisting}

\passthrough{\lstinline!q!} stands for ``quantity'', which is what we
are looking up. \passthrough{\lstinline!p!} stands for probability,
which is the result. In this example, the quantity is age
\passthrough{\lstinline!51!}, and the corresponding probability is about
\passthrough{\lstinline!0.62!}. That means that about \(62\%\) of the
respondents are \passthrough{\lstinline!51!} years old or younger. The
arrow in the following figure shows how you could read this value from
the CDF, at least approximately.

\begin{lstlisting}[language=Python,style=source]
cdf_age.plot()

x = 17
draw_line(p, q, x)
draw_arrow_left(p, q, x)

plt.xlabel('Age (years)')
plt.xlim(x-1, 91)
plt.ylabel('CDF')
plt.title('Distribution of age');
\end{lstlisting}

\begin{center}
\includegraphics[width=4in]{chapters/08_distributions_files/08_distributions_66_0.png}
\end{center}

The CDF is an invertible function, which means that if you have a
probability, \passthrough{\lstinline!p!}, you can look up the
corresponding quantity, \passthrough{\lstinline!q!}.
\passthrough{\lstinline!Cdf!} provides a method called
\passthrough{\lstinline!inverse!} that computes the inverse of the
cumulative distribution function.

\begin{lstlisting}[language=Python,style=source]
p1 = 0.25
q1 = cdf_age.inverse(p1)
q1
\end{lstlisting}

\begin{lstlisting}[style=output]
array(32.)
\end{lstlisting}

In this example, we look up the probability
\passthrough{\lstinline!0.25!} and the result is
\passthrough{\lstinline!32!}.\\
That means that 25\% of the respondents are age 32 or less. Another way
to say the same thing is ``age 32 is the 25th percentile of this
distribution''.

If we look up probability \passthrough{\lstinline!0.75!}, it returns
\passthrough{\lstinline!60!}, so 75\% of the respondents are 60 or
younger.

\begin{lstlisting}[language=Python,style=source]
p2 = 0.75
q2 = cdf_age.inverse(p2)
q2
\end{lstlisting}

\begin{lstlisting}[style=output]
array(60.)
\end{lstlisting}

In the following figure, the arrows show how you could read these values
from the CDF.

\begin{lstlisting}[language=Python,style=source]
cdf_age.plot()

x = 17
draw_line(p1, q1, x)
draw_arrow_down(p1, q1, 0)

draw_line(p2, q2, x)
draw_arrow_down(p2, q2, 0)

plt.xlabel('Age (years)')
plt.xlim(x-1, 91)
plt.ylabel('CDF')
plt.title('Distribution of age');
\end{lstlisting}

\begin{center}
\includegraphics[width=4in]{chapters/08_distributions_files/08_distributions_72_0.png}
\end{center}

The distance from the 25th to the 75th percentile is called the
\textbf{interquartile range}, or IQR. It measures the spread of the
distribution, so it is similar to standard deviation or variance.
Because it is based on percentiles, it doesn't get thrown off by extreme
values or outliers, the way standard deviation does. So IQR is more
\textbf{robust} than variance, which means it works well even if there
are errors in the data or extreme values.

\textbf{Exercise:} Using \passthrough{\lstinline!cdf\_age!}, compute the
fraction of respondents in the GSS dataset who are \emph{older} than
\passthrough{\lstinline!65!}. Recall that the CDF computes the fraction
who are less than or equal to a value, so the complement is the fraction
who exceed a value.

\textbf{Exercise:} The distribution of income in almost every country is
long-tailed, which means there are a small number of people with very
high incomes. In the GSS dataset, the column
\passthrough{\lstinline!realinc!} represents total household income,
converted to 1986 dollars. We can get a sense of the shape of this
distribution by plotting the CDF. Select
\passthrough{\lstinline!realinc!} from the \passthrough{\lstinline!gss!}
dataset, make a \passthrough{\lstinline!Cdf!} called
\passthrough{\lstinline!cdf\_income!}, and plot it. Remember to label
the axes!

Because the tail of the distribution extends to the right, the mean is
greater than the median. Use the \passthrough{\lstinline!Cdf!} object to
compute the fraction of respondents whose income is below the mean.

\hypertarget{comparing-distributions}{%
\section{Comparing Distributions}\label{comparing-distributions}}

So far we've seen two ways to represent distributions, PMFs and CDFs.
Now we'll use PMFs and CDFs to compare distributions, and we'll see the
pros and cons of each. One way to compare distributions is to plot
multiple PMFs on the same axes. For example, suppose we want to compare
the distribution of age for male and female respondents. First we'll
create a Boolean Series that's true for male respondents and another
that's true for female respondents.

\begin{lstlisting}[language=Python,style=source]
male = (gss['sex'] == 1)
female = (gss['sex'] == 2)
\end{lstlisting}

Now we can select ages for the male and female respondents.

\begin{lstlisting}[language=Python,style=source]
male_age = age[male]
female_age = age[female]
\end{lstlisting}

And plot a PMF for each.

\begin{lstlisting}[language=Python,style=source]
pmf_male_age = Pmf.from_seq(male_age)
pmf_male_age.plot(label='Male')

pmf_female_age = Pmf.from_seq(female_age)
pmf_female_age.plot(label='Female')

plt.xlabel('Age (years)') 
plt.ylabel('PMF')
plt.title('Distribution of age by sex')
plt.legend();
\end{lstlisting}

\begin{center}
\includegraphics[width=4in]{chapters/08_distributions_files/08_distributions_81_0.png}
\end{center}

A plot like this, which is highly variable, is often described as
\textbf{noisy}. If we ignore the noise, it looks like the PMF is higher
for men between ages 40 and 50, and higher for women between ages 70 and
80. But both of those differences might be due to random variation.

Now let's do the same thing with CDFs -- everything is the same except
we replace \passthrough{\lstinline!Pmf!} with
\passthrough{\lstinline!Cdf!}.

\begin{lstlisting}[language=Python,style=source]
cdf_male_age = Cdf.from_seq(male_age)
cdf_male_age.plot(label='Male')

cdf_female_age = Cdf.from_seq(female_age)
cdf_female_age.plot(label='Female')

plt.xlabel('Age (years)') 
plt.ylabel('CDF')
plt.title('Distribution of age by sex')
plt.legend();
\end{lstlisting}

\begin{center}
\includegraphics[width=4in]{chapters/08_distributions_files/08_distributions_83_0.png}
\end{center}

In general, CDFs are smoother than PMFs. Because they smooth out
randomness, we can often get a better view of real differences between
distributions. In this case, the lines are close together until age 40
-- after that, the CDF is higher for men than women.

So what does that mean? One way to interpret the difference is that the
fraction of men below a given age is generally more than the fraction of
women below the same age. For example, about 79\% of men are
\passthrough{\lstinline!60!} or less, compared to 76\% of women.

\begin{lstlisting}[language=Python,style=source]
cdf_male_age(60), cdf_female_age(60)
\end{lstlisting}

\begin{lstlisting}[style=output]
(array(0.7721998), array(0.7474241))
\end{lstlisting}

Going the other way, we could also compare percentiles. For example, the
median age woman is older than the median age man, by about one year.

\begin{lstlisting}[language=Python,style=source]
cdf_male_age.inverse(0.5), cdf_female_age.inverse(0.5)
\end{lstlisting}

\begin{lstlisting}[style=output]
(array(44.), array(45.))
\end{lstlisting}

\textbf{Exercise:} What fraction of men are over 80? What fraction of
women?

\hypertarget{comparing-incomes}{%
\section{Comparing Incomes}\label{comparing-incomes}}

As another example, let's look at household income and compare the
distribution before and after 1995 (I chose 1995 because it's roughly
the midpoint of the survey). The column
\passthrough{\lstinline!realinc!} represents household income in 1986
dollars. We'll make two Boolean \passthrough{\lstinline!Series!} objects
to select respondents interviewed before and after 1995.

\begin{lstlisting}[language=Python,style=source]
pre95 = (gss['year'] < 1995)
post95 = (gss['year'] >= 1995)
\end{lstlisting}

Now we can plot the PMFs.

\begin{lstlisting}[language=Python,style=source]
income = gss['realinc'].replace(0, np.nan)

Pmf.from_seq(income[pre95]).plot(label='Before 1995')
Pmf.from_seq(income[post95]).plot(label='After 1995')

plt.xlabel('Income (1986 USD)')
plt.ylabel('PMF')
plt.title('Distribution of income')
plt.legend();
\end{lstlisting}

\begin{center}
\includegraphics[width=4in]{chapters/08_distributions_files/08_distributions_92_0.png}
\end{center}

There are a lot of unique values in this distribution, and none of them
appear very often. As a result, the PMF is so noisy and we can't really
see the shape of the distribution. It's also hard to compare the
distributions. It looks like there are more people with high incomes
after 1995, but it's hard to tell. We can get a clearer picture with a
CDF.

\begin{lstlisting}[language=Python,style=source]
Cdf.from_seq(income[pre95]).plot(label='Before 1995')
Cdf.from_seq(income[post95]).plot(label='After 1995')

plt.xlabel('Income (1986 USD)')
plt.ylabel('CDF')
plt.title('Distribution of income')
plt.legend();
\end{lstlisting}

\begin{center}
\includegraphics[width=4in]{chapters/08_distributions_files/08_distributions_94_0.png}
\end{center}

Below \$30,000 the CDFs are almost identical; above that, we can see
that the post-1995 distribution is shifted to the right. In other words,
the fraction of people with high incomes is about the same, but the
income of high earners has increased.

In general, I recommend CDFs for exploratory analysis. They give you a
clear view of the distribution, without too much noise, and they are
good for comparing distributions -- especially if you have more than
two.

\textbf{Exercise:} Let's compare incomes for different levels of
education in the GSS dataset. We'll use the
\passthrough{\lstinline!degree!} column, which represents the highest
degree each respondent has earned. In this column, the value
\passthrough{\lstinline!1!} indicates a high school diploma,
\passthrough{\lstinline!2!} indicates an Associate's degree, and
\passthrough{\lstinline!3!} indicates a Bachelor's degree.

Compute and plot the distribution of income for each group. Remember to
label the CDFs, display a legend, and label the axes. Write a few
sentences that describe and interpret the results.

\hypertarget{modeling-distributions}{%
\section{Modeling Distributions}\label{modeling-distributions}}

Some distributions have names. For example, you might be familiar with
the normal distribution, also called the Gaussian distribution or the
bell curve. And you might have heard of others like the exponential
distribution, binomial distribution, or maybe Poisson distribution.
These ``distributions with names'' are called \textbf{analytic} because
they are described by analytic mathematical functions, as contrasted
with empirical distributions, which are based on data.

It turns out that many things we measure have distributions that are
well approximated by analytic distributions, so these distributions are
sometimes good models for the real world. In this context, what I mean
by a \textbf{model} is a simplified description of the world that is
accurate enough for its intended purpose. In this section, we'll compute
the CDF of a normal distribution and compare it to an empirical
distribution of data. But before we get to real data, we'll start with
fake data.

The following statement uses NumPy's \passthrough{\lstinline!random!}
library to generate 1000 values from a normal distribution with mean
\passthrough{\lstinline!0!} and standard deviation
\passthrough{\lstinline!1!}.

\begin{lstlisting}[language=Python,style=source]
sample = np.random.normal(size=1000)
\end{lstlisting}

Here's what the empirical distribution of the sample looks like.

\begin{lstlisting}[language=Python,style=source]
cdf_sample = Cdf.from_seq(sample)
cdf_sample.plot(label='Random sample')

plt.xlabel('x')
plt.ylabel('CDF')
plt.legend();
\end{lstlisting}

\begin{center}
\includegraphics[width=4in]{chapters/08_distributions_files/08_distributions_102_0.png}
\end{center}

If we did not know that this sample was drawn from a normal
distribution, and we wanted to check, we could compare the CDF of the
data to the CDF of an ideal normal distribution, which we can use the
SciPy library to compute.

\begin{lstlisting}[language=Python,style=source]
from scipy.stats import norm

xs = np.linspace(-3, 3)
ys = norm(0, 1).cdf(xs)
\end{lstlisting}

First we import \passthrough{\lstinline!norm!} from
\passthrough{\lstinline!scipy.stats!}, which is a collection of
functions related to statistics. Then we use
\passthrough{\lstinline!linspace()!} to create an array of
equally-spaced points from -3 to 3; those are the
\passthrough{\lstinline!x!} values where we will evaluate the normal
CDF. Next, \passthrough{\lstinline!norm(0, 1)!} creates an object that
represents a normal distribution with mean \passthrough{\lstinline!0!}
and standard deviation \passthrough{\lstinline!1!}. Finally,
\passthrough{\lstinline!cdf!} computes the CDF of the normal
distribution, evaluated at each of the \passthrough{\lstinline!xs!}.
I'll plot the normal CDF with a gray line and then plot the CDF of the
data again.

\begin{lstlisting}[language=Python,style=source]
plt.plot(xs, ys, color='gray', label='Normal CDF')
cdf_sample.plot(label='Random sample')

plt.xlabel('x')
plt.ylabel('CDF')
plt.legend();
\end{lstlisting}

\begin{center}
\includegraphics[width=4in]{chapters/08_distributions_files/08_distributions_106_0.png}
\end{center}

The CDF of the random sample agrees with the normal model -- which is
not surprising because the data were actually sampled from a normal
distribution. When we collect data in the real world, we do not expect
it to fit a normal distribution as well as this. In the next exercise,
we'll try it and see.

\textbf{Exercise:} In many datasets, the distribution of income is
approximately \textbf{lognormal}, which means that the logarithms of the
incomes fit a normal distribution. Let's see whether that's true for the
GSS data.

\begin{itemize}
\item
  Extract \passthrough{\lstinline!realinc!} from
  \passthrough{\lstinline!gss!} and compute its logarithm using
  \passthrough{\lstinline!np.log10()!}.
\item
  Compute the mean and standard deviation of the log-transformed
  incomes.
\item
  Use \passthrough{\lstinline!norm!} to make a normal distribution with
  the same mean and standard deviation as the log-transformed incomes.
\item
  Plot the CDF of the normal distribution.
\item
  Compute and plot the CDF of the log-transformed incomes.
\end{itemize}

How similar are the CDFs of the log-transformed incomes and the normal
distribution?

\hypertarget{kernel-density-estimation}{%
\section{Kernel Density Estimation}\label{kernel-density-estimation}}

We have seen two ways to represent distributions, PMFs and CDFs. Now
we'll learn another way: a probability density function, or PDF. The
\passthrough{\lstinline!norm!} function, which we used to compute the
normal CDF, can also compute the normal PDF.

\begin{lstlisting}[language=Python,style=source]
xs = np.linspace(-3, 3)
ys = norm(0,1).pdf(xs)
plt.plot(xs, ys, color='gray', label='Normal PDF')

plt.xlabel('x')
plt.ylabel('PDF')
plt.title('Normal density function')
plt.legend();
\end{lstlisting}

\begin{center}
\includegraphics[width=4in]{chapters/08_distributions_files/08_distributions_112_0.png}
\end{center}

The normal PDF is the classic ``bell curve''. It is tempting to compare
the PMF of the data to the PDF of the normal distribution, but that
doesn't work. Let's see what happens if we try:

\begin{lstlisting}[language=Python,style=source]
plt.plot(xs, ys, color='gray', label='Normal PDF')

pmf_sample = Pmf.from_seq(sample)
pmf_sample.plot(label='Random sample')

plt.xlabel('x')
plt.ylabel('PDF')
plt.title('Normal density function')
plt.legend();
\end{lstlisting}

\begin{center}
\includegraphics[width=4in]{chapters/08_distributions_files/08_distributions_114_0.png}
\end{center}

The PMF of the sample is a flat line across the bottom. In the random
sample, every value is unique, so they all have the same probability,
one in 1000. However, we can use the points in the sample to estimate
the PDF of the distribution they came from. This process is called
\textbf{kernel density estimation}, or KDE. It's a way of getting from a
PMF, a probability mass function, to a PDF, a probability density
function.

To generate a KDE plot, we'll use the Seaborn library, imported as
\passthrough{\lstinline!sns!}. Seaborn provides
\passthrough{\lstinline!kdeplot!}, which takes the sample, estimates the
PDF, and plots it.

\begin{lstlisting}[language=Python,style=source]
import seaborn as sns

sns.kdeplot(sample, label='Estimated sample PDF')

plt.xlabel('x')
plt.ylabel('PDF')
plt.title('Normal density function')
plt.legend();
\end{lstlisting}

\begin{center}
\includegraphics[width=4in]{chapters/08_distributions_files/08_distributions_116_0.png}
\end{center}

Now we can compare the KDE plot and the normal PDF.

\begin{lstlisting}[language=Python,style=source]
plt.plot(xs, ys, color='gray', label='Normal PDF')
sns.kdeplot(sample, label='Estimated sample PDF')

plt.xlabel('x')
plt.ylabel('PDF')
plt.title('Normal density function')
plt.legend();
\end{lstlisting}

\begin{center}
\includegraphics[width=4in]{chapters/08_distributions_files/08_distributions_118_0.png}
\end{center}

The KDE plot matches the normal PDF pretty well, although the
differences look bigger when we compare PDFs than they did with the
CDFs. That means that the PDF is a more sensitive way to look for
differences, but often it is too sensitive.\\
It's hard to tell whether apparent differences mean anything, or if they
are just random, as in this case.

\textbf{Exercise:} In a previous exercise, we used CDFs to see if the
distribution of income fits a lognormal distribution. We can make the
same comparison using a PDF and KDE.

\begin{itemize}
\item
  Again, extract \passthrough{\lstinline!realinc!} from
  \passthrough{\lstinline!gss!} and compute its logarithm using
  \passthrough{\lstinline!np.log10()!}.
\item
  Compute the mean and standard deviation of the log-transformed
  incomes.
\item
  Use \passthrough{\lstinline!norm!} to make a normal distribution with
  the same mean and standard deviation as the log-transformed incomes.
\item
  Plot the PDF of the normal distribution.
\item
  Use \passthrough{\lstinline!sns.kdeplot()!} to estimate and plot the
  density of the log-transformed incomes.
\end{itemize}

\hypertarget{summary}{%
\section{Summary}\label{summary}}

In this chapter, we've seen four ways to visualize distributions, PMFs,
CDFs, and KDE plots. In general, I use CDFs when I am exploring data.
That way, I get the best view of what's going on without getting
distracted by noise. Then, if I am presenting results to an audience
unfamiliar with CDFs, I might use a PMF if the dataset contains a small
number of unique values, or KDE if there are many unique values.

