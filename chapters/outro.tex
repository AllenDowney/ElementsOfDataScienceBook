\hypertarget{further-reading}{%
\chapter{Further Reading}\label{further-reading}}

The first part of this book is an accelerated introduction to Python
with emphasis on tools for working with data. One of the benefits of
learning Python is that it useful for many other kinds of computing, not
just data science. If you would like to learn more about Python, there
are a lot of good books and online resources, but if the style of this
book works well for you, you might like \emph{Think Python}, also by
Allen Downey and published by O'Reilly Media.

If you are interested in scientific computing, you might like
\emph{Modeling and Simulation in Python}, published by No Starch Press.
It is an introduction to Python focused on modeling and simulating
physical systems. It explores a range of topics including population
growth, thermodynamics, and simple mechanical systems.

The second part of this book is about exploratory data analysis and
visualization. If you are interested in data visualization, you might
like Nathan Yau's blog, \emph{FlowingData}, and his books,
\emph{Visualize This} and \emph{Data Points}.

The third part of this book is about statistical inference, that is,
using data from a sample to estimate something about a population. We
used resampling to quantify the precision of those estimates, and
hypothesis testing to consider whether an effect we observe might be due
to chance. The process I demonstrated might be called conventional
inference, in contrast to the alternative, which is Bayesian inference.
If you are interested in learning more about that, you might like
\emph{Think Bayes}, Also by Allen Downey and published by O'Reilly
Media.

The Political Alignment case study uses data from the General Social
Survey (GSS) to explore political beliefs in the United States, how they
differ between groups, and how they change over time. If you are
interested in this topic, you might like \emph{Probably Overthinking
It}, forthcoming from University of Chicago Press, which explores the
GSS data in greater depth. It presents a variety of other topics as
well, exploring, as the subtitle explains, ``How to use data to answer
questions, avoid statistical traps, and make better decisions''.

Finally, the Recidivism case study explores the use of predictive
algorithms in the criminal justice system and explains the metrics we
use to assess them. There are many books and articles on this topic,
which reflects its importance because of the impact it has on people's
lives, and also the difficulty of resolving conflicting requirements of
fairness. If you would like to read more on this topic, I recommend Orly
Lobel's recent book, \emph{The Equality Machine}, which reviews many of
the challenges algorithms pose while also recognizing their potential to
do good.



