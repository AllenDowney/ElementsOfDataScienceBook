\chapter{Dictionaries}\label{dictionaries}

In the previous chapter we used a \passthrough{\lstinline!for!} loop to
read a file and count the words. In this chapter we'll count the number
of \emph{unique} words and the number of times each one appears. To do
that, we'll use one of Python's most useful features, a
\textbf{dictionary}.

You will also see how to select an element from a sequence (tuple, list,
or array). And you will learn a little about Unicode, which is used to
represent letters, numbers, and punctuation for almost every language in
the world.

\section{Indexing}\label{indexing}

Suppose you have a variable named \passthrough{\lstinline!t!} that
refers to a list or tuple. You can select an element using the
\textbf{bracket operator}, \passthrough{\lstinline![]!}. For example,
here's a tuple of strings:

\begin{lstlisting}[language=Python,style=source]
t = ('zero', 'one', 'two')
\end{lstlisting}

To select the first element, we put \passthrough{\lstinline!0!} in
brackets:

\begin{lstlisting}[language=Python,style=source]
t[0]
\end{lstlisting}

\begin{lstlisting}[style=output]
'zero'
\end{lstlisting}

\pagebreak

To select the second element, we put \passthrough{\lstinline!1!} in
brackets:

\begin{lstlisting}[language=Python,style=source]
t[1]
\end{lstlisting}

\begin{lstlisting}[style=output]
'one'
\end{lstlisting}

To select the third element, we put \passthrough{\lstinline!2!} in
brackets:

\begin{lstlisting}[language=Python,style=source]
t[2]
\end{lstlisting}

\begin{lstlisting}[style=output]
'two'
\end{lstlisting}

The number in brackets is called an \textbf{index} because it indicates
which element we want. Tuples and lists use zero-based numbering -- that
is, the index of the first element is 0. Some other programming
languages use one-based numbering.

The index in brackets can also be a variable:

\begin{lstlisting}[language=Python,style=source]
i = 1
t[i]
\end{lstlisting}

\begin{lstlisting}[style=output]
'one'
\end{lstlisting}

Or an expression with variables, values, and operators:

\begin{lstlisting}[language=Python,style=source]
t[i+1]
\end{lstlisting}

\begin{lstlisting}[style=output]
'two'
\end{lstlisting}

But if the index goes past the end of the sequence, you get an error.

\begin{lstlisting}[language=Python,style=source]
%%expect IndexError

t[3]
\end{lstlisting}

\begin{lstlisting}[style=output]
IndexError: tuple index out of range
\end{lstlisting}

Also, the index has to be an integer -- if it is any other type, you get
an error.

\begin{lstlisting}[language=Python,style=source]
%%expect TypeError

t[1.5]
\end{lstlisting}

\begin{lstlisting}[style=output]
TypeError: tuple indices must be integers or slices, not float
\end{lstlisting}

\begin{lstlisting}[language=Python,style=source]
%%expect TypeError

t['1']
\end{lstlisting}

\begin{lstlisting}[style=output]
TypeError: tuple indices must be integers or slices, not str
\end{lstlisting}

\textbf{Exercise:} You can use negative integers as indices. Try using
\passthrough{\lstinline!-1!} and \passthrough{\lstinline!-2!} as
indices, and see if you can figure out what they do.

\section{Dictionaries}\label{dictionaries-1}

A dictionary is similar to a tuple or list, but in a dictionary, the
index can be almost any type, not just an integer. We can create an
empty dictionary like this:

\begin{lstlisting}[language=Python,style=source]
d = {}
\end{lstlisting}

Then we can add elements like this:

\begin{lstlisting}[language=Python,style=source]
d['one'] = 1
d['two'] = 2
\end{lstlisting}

In this example, the indices are the strings,
\passthrough{\lstinline!'one'!} and \passthrough{\lstinline!'two'!}. If
you display the dictionary, it shows each index and the corresponding
value.

\begin{lstlisting}[language=Python,style=source]
d
\end{lstlisting}

\begin{lstlisting}[style=output]
{'one': 1, 'two': 2}
\end{lstlisting}

Instead of creating an empty dictionary and then adding elements, you
can create a dictionary and specify the elements at the same time:

\begin{lstlisting}[language=Python,style=source]
d = {'one': 1, 'two': 2, 'three': 3}
d
\end{lstlisting}

\begin{lstlisting}[style=output]
{'one': 1, 'two': 2, 'three': 3}
\end{lstlisting}

When we are talking about dictionaries, an index is usually called a
\textbf{key}. In this example, the keys are strings and the
corresponding values are integers. A dictionary is also called a
\textbf{map}, because it represents a correspondence or ``mapping'',
between keys and values. So we might say that this dictionary maps from
English number names to the corresponding integers.

\pagebreak

You can use the bracket operator to select an element from a dictionary,
like this:

\begin{lstlisting}[language=Python,style=source]
d['two']
\end{lstlisting}

\begin{lstlisting}[style=output]
2
\end{lstlisting}

But don't forget the quotation marks. Without them, Python looks for a
variable named \passthrough{\lstinline!two!} and doesn't find one.

\begin{lstlisting}[language=Python,style=source]
%%expect NameError

d[two]
\end{lstlisting}

\begin{lstlisting}[style=output]
NameError: name 'two' is not defined
\end{lstlisting}

To check whether a particular key is in a dictionary, you can use the
\passthrough{\lstinline!in!} operator:

\begin{lstlisting}[language=Python,style=source]
'one' in d
\end{lstlisting}

\begin{lstlisting}[style=output]
True
\end{lstlisting}

\begin{lstlisting}[language=Python,style=source]
'zero' in d
\end{lstlisting}

\begin{lstlisting}[style=output]
False
\end{lstlisting}

Because the word \passthrough{\lstinline!in!} is an operator in Python,
you can't use it as a variable name.

\begin{lstlisting}[language=Python,style=source]
%%expect SyntaxError

in = 5
\end{lstlisting}

\begin{lstlisting}[style=output]
  Cell In[22], line 1
    in = 5
    ^
SyntaxError: invalid syntax
\end{lstlisting}

Each key in a dictionary can only appear once. Adding the same key again
has no effect:

\begin{lstlisting}[language=Python,style=source]
d['one'] = 1
d
\end{lstlisting}

\begin{lstlisting}[style=output]
{'one': 1, 'two': 2, 'three': 3}
\end{lstlisting}

\pagebreak

But you can change the value associated with a key:

\begin{lstlisting}[language=Python,style=source]
d['one'] = 100
d
\end{lstlisting}

\begin{lstlisting}[style=output]
{'one': 100, 'two': 2, 'three': 3}
\end{lstlisting}

You can loop through the keys in a dictionary like this:

\begin{lstlisting}[language=Python,style=source]
for key in d:
    print(key)
\end{lstlisting}

\begin{lstlisting}[style=output]
one
two
three
\end{lstlisting}

If you want the keys and the values, one way to get them is to loop
through the keys and look up the values:

\begin{lstlisting}[language=Python,style=source]
for key in d:
    print(key, d[key])
\end{lstlisting}

\begin{lstlisting}[style=output]
one 100
two 2
three 3
\end{lstlisting}

Or you can loop through both at the same time, like this:

\begin{lstlisting}[language=Python,style=source]
for key, value in d.items():
    print(key, value)
\end{lstlisting}

\begin{lstlisting}[style=output]
one 100
two 2
three 3
\end{lstlisting}

The \passthrough{\lstinline!items!} method loops through the key-value
pairs in the dictionary. Each time through the loop, they are assigned
to \passthrough{\lstinline!key!} and \passthrough{\lstinline!value!}.

\textbf{Exercise:} Make a dictionary with the integers
\passthrough{\lstinline!1!}, \passthrough{\lstinline!2!}, and
\passthrough{\lstinline!3!} as keys and strings as values. The strings
should be the words ``one'', ``two'', and ``three'' or their equivalents
in any language you know.

Write a loop that prints just the values from the dictionary.

\section{Counting Unique Words}\label{counting-unique-words}

In the previous chapter we downloaded \emph{War and Peace} from Project
Gutenberg and counted the number of lines and words. Now that we have
dictionaries, we can also count the number of unique words and the
number of times each one appears.

As we did in the previous chapter, we can read the text of \emph{War and
Peace} and count the number of words.

\begin{lstlisting}[language=Python,style=source]
fp = open('2600-0.txt')
count = 0
for line in fp:
    count += len(line.split())

count
\end{lstlisting}

\begin{lstlisting}[style=output]
566316
\end{lstlisting}

To count the number of unique words, we'll loop through the words in
each line and add them as keys in a dictionary:

\begin{lstlisting}[language=Python,style=source]
fp = open('2600-0.txt')
unique_words = {}
for line in fp:
    for word in line.split():
        unique_words[word] = 1
\end{lstlisting}

This is the first example we've seen with one loop \textbf{nested}
inside another.

\begin{itemize}
\item
  The outer loop runs through the lines in the file.
\item
  The inner loops runs through the words in each line.
\end{itemize}

Each time through the inner loop, we add a word as a key in the
dictionary, with the value \passthrough{\lstinline!1!}. If a word that
is already in the dictionary appears again, adding it to the dictionary
again has no effect. So the dictionary gets only one copy of each unique
word in the file. At the end of the loop, we can display the first eight
keys like this.

\begin{lstlisting}[language=Python,style=source]
list(unique_words)[:8]
\end{lstlisting}

\begin{lstlisting}[style=output]
['The', 'Project', 'Gutenberg', 'EBook', 'of', 'War', 'and', 'Peace,']
\end{lstlisting}

The \passthrough{\lstinline!list!} function puts the keys from the
dictionary in a list. In the bracket operator,
\passthrough{\lstinline!:8!} is a special index called a \emph{slice}
that selects the first eight elements.

Each word only appears once, so the number of keys is the number of
unique words.

\begin{lstlisting}[language=Python,style=source]
len(unique_words)
\end{lstlisting}

\begin{lstlisting}[style=output]
41990
\end{lstlisting}

There are about 42,000 different words in the book, which is
substantially less than the total number of words, about 560,000. But
this count is not correct yet, because we have not taken into account
capitalization and punctuation.

\textbf{Exercise:} Before we deal with those problems, let's practice
with nested loops -- that is, one loop inside another. Suppose you have
a list of words, like this:

\begin{lstlisting}[language=Python,style=source]
line = ['War', 'and', 'Peace']
\end{lstlisting}

Write a nested loop that iterates through each word in the list, and
each letter in each word, and prints the letters on separate lines.

\section{Dealing with Capitalization}\label{dealing-with-capitalization}

When we count unique words, we probably want to treat
\passthrough{\lstinline!The!} and \passthrough{\lstinline!the!} as the
same word. We can do that by converting all words to lower case, using
the \passthrough{\lstinline!lower!} function:

\begin{lstlisting}[language=Python,style=source]
word = 'The'
word.lower()
\end{lstlisting}

\begin{lstlisting}[style=output]
'the'
\end{lstlisting}

\passthrough{\lstinline!lower!} creates a new string; it does not modify
the original string.

\begin{lstlisting}[language=Python,style=source]
word
\end{lstlisting}

\begin{lstlisting}[style=output]
'The'
\end{lstlisting}

However, you can assign the new string back to the existing variable,
like this:

\begin{lstlisting}[language=Python,style=source]
word = word.lower()
\end{lstlisting}

Now if we can display the new value of \passthrough{\lstinline!word!},
we get the lowercase version:

\begin{lstlisting}[language=Python,style=source]
word
\end{lstlisting}

\begin{lstlisting}[style=output]
'the'
\end{lstlisting}

\textbf{Exercise:} Modify the previous loop so it makes a lowercase
version of each word before adding it to the dictionary. How many unique
words are there, if we ignore the difference between uppercase and
lowercase?

\section{Removing Punctuation}\label{removing-punctuation}

To remove punctuation from the words, we can use
\passthrough{\lstinline!strip!}, which removes characters from the
beginning and end of a string. Here's an example:

\begin{lstlisting}[language=Python,style=source]
word = 'abracadabra'
word.strip('ab')
\end{lstlisting}

\begin{lstlisting}[style=output]
'racadabr'
\end{lstlisting}

In this example, \passthrough{\lstinline!strip!} removes all instances
of \passthrough{\lstinline!a!} and \passthrough{\lstinline!b!} from the
beginning and end of the word, but not from the middle. Like
\passthrough{\lstinline!lower!}, this function makes a new word -- it
doesn't modify the original:

\begin{lstlisting}[language=Python,style=source]
word
\end{lstlisting}

\begin{lstlisting}[style=output]
'abracadabra'
\end{lstlisting}

To remove punctuation, we can use the \passthrough{\lstinline!string!}
library, which provides a variable named
\passthrough{\lstinline!punctuation!}.

\begin{lstlisting}[language=Python,style=source]
import string

string.punctuation
\end{lstlisting}

\begin{lstlisting}[style=output]
'!"#$%&\'()*+,-./:;<=>?@[\\]^_`{|}~'
\end{lstlisting}

\passthrough{\lstinline!string.punctuation!} contains the most common
punctuation marks, but as we'll see, not all of them. Nevertheless, we
can use it to handle most cases. Here's an example:

\begin{lstlisting}[language=Python,style=source]
line = "It's not given to people to judge what's right or wrong."

for word in line.split():
    word = word.strip(string.punctuation)
    print(word)
\end{lstlisting}

\begin{lstlisting}[style=output]
It's
not
given
to
people
to
judge
what's
right
or
wrong
\end{lstlisting}

\passthrough{\lstinline!strip!} removes the period at the end of
\passthrough{\lstinline!wrong!}, but not the apostrophes in
\passthrough{\lstinline!It's!}, \passthrough{\lstinline!don't!} and
\passthrough{\lstinline!what's!}. That's good, because we want to treat
an apostrophe as part of a word.
But we have one more problem to solve. Here's another line from the book.

\begin{lstlisting}[language=Python,style=source]
line = 'anyone, and so you don't deserve to have them.”'
\end{lstlisting}

Here's what happens when we try to remove the punctuation.

\begin{lstlisting}[language=Python,style=source]
for word in line.split():
    word = word.strip(string.punctuation)
    print(word)
\end{lstlisting}

\begin{lstlisting}[style=output]
anyone
and
so
you
don't
deserve
to
have
them.”
\end{lstlisting}

The comma after \passthrough{\lstinline!anyone!} is removed, but not the
quotation mark at the end of the last word.
The problem is that this kind of quotation mark is not in
\passthrough{\lstinline!string.punctuation!}, so
\passthrough{\lstinline!strip!} doesn't remove it. To fix this problem,
we'll use the following loop, which

\begin{enumerate}
\def\labelenumi{\arabic{enumi}.}
\item
  Reads the file and builds a dictionary that contains all punctuation
  marks that appear in the book, then
\item
  It uses the \passthrough{\lstinline!join!} function to concatenate the
  keys of the dictionary in a single string.
\end{enumerate}

You don't have to understand everything about how it works, but I
suggest you read it and see how much you can figure out.

\begin{lstlisting}[language=Python,style=source]
import unicodedata

fp = open('2600-0.txt')
punc_marks = {}
for line in fp:
    for x in line:
        category = unicodedata.category(x)
        if category[0] == 'P':
            punc_marks[x] = 1

all_punctuation = ''.join(punc_marks)
\end{lstlisting}

\pagebreak

\begin{lstlisting}[language=Python,style=source]
print(all_punctuation)
\end{lstlisting}

\begin{lstlisting}[style=output]
,.-:[#]*/“’—‘!?”;()%@
\end{lstlisting}

The result is a string containing all of the punctuation characters that
appear in the document, in the order they first appear.

\textbf{Exercise:} Modify the word-counting loop from the previous
section to convert words to lower case \emph{and} strip punctuation
before adding them to the dictionary. Now how many unique words are
there?

\section{Counting Word Frequencies}\label{counting-word-frequencies}

In the previous section we counted the number of unique words, but we
might also want to know how often each word appears. Then we can find
the most common and least common words in the book. To count the
frequency of each word, we'll make a dictionary that maps from each word
to the number of times it appears.

Here's an example that loops through a string and counts the number of
times each letter appears.

\begin{lstlisting}[language=Python,style=source]
word = 'Mississippi'

letter_counts = {}
for x in word:
    if x in letter_counts:
        letter_counts[x] += 1
    else:
        letter_counts[x] = 1

letter_counts
\end{lstlisting}

\begin{lstlisting}[style=output]
{'M': 1, 'i': 4, 's': 4, 'p': 2}
\end{lstlisting}

The \passthrough{\lstinline!if!} statement includes a feature we have
not seen before, an \passthrough{\lstinline!else!} clause. Here's how it
works.

\begin{enumerate}
\def\labelenumi{\arabic{enumi}.}
\item
  First, it checks whether the letter, \passthrough{\lstinline!x!}, is
  already a key in the dictionary,
  \passthrough{\lstinline!letter\_counts!}.
\item
  If so, it runs the first statement,
  \passthrough{\lstinline!letter\_counts[x] += 1!}, which increments the
  value associated with the letter.
\item
  Otherwise, it runs the second statement,
  \passthrough{\lstinline!letter\_counts[x] = 1!}, which adds
  \passthrough{\lstinline!x!} as a new key, with the value
  \passthrough{\lstinline!1!} indicating that we have seen the new
  letter once.
\end{enumerate}

The result is a dictionary that maps from each letter to the number of
times it appears. To get the most common letters, we can use a
\passthrough{\lstinline!Counter!}, which is similar to a dictionary. To
use it, we have to import a library called
\passthrough{\lstinline!collections!}:

\begin{lstlisting}[language=Python,style=source]
import collections
\end{lstlisting}

Then we use \passthrough{\lstinline!collections.Counter!} to convert the
dictionary to a \passthrough{\lstinline!Counter!}:

\begin{lstlisting}[language=Python,style=source]
counter = collections.Counter(letter_counts)
type(counter)
\end{lstlisting}

\begin{lstlisting}[style=output]
collections.Counter
\end{lstlisting}

\passthrough{\lstinline!Counter!} provides a function called
\passthrough{\lstinline!most\_common!} we can use to get the most common
characters:

\begin{lstlisting}[language=Python,style=source]
counter.most_common(3)
\end{lstlisting}

\begin{lstlisting}[style=output]
[('i', 4), ('s', 4), ('p', 2)]
\end{lstlisting}

The result is a list of tuples, where each tuple contains a character
and count, sorted by count.

\textbf{Exercise:} Modify the loop from the previous exercise to count
the frequency of the words in \emph{War and Peace}. Then print the 20
most common words and the number of times each one appears.

\textbf{Exercise:} You can run \passthrough{\lstinline!most\_common!}
with no value in parentheses, like this:

\begin{lstlisting}[style=output]
word_freq_pairs = counter.most_common()
\end{lstlisting}

The result is a list of tuples, with one tuple for every unique word in
the book. Use it to answer the following questions:

\begin{enumerate}
\def\labelenumi{\arabic{enumi}.}
\item
  How many times does the \#1 ranked word appear (that is, the first
  element of the list)?
\item
  How many times does the \#10 ranked word appear?
\item
  How many times does the \#100 ranked word appear?
\item
  How many times does the \#1000 ranked word appear?
\item
  How many times does the \#10000 ranked word appear?
\end{enumerate}

Do you see a pattern in the results? We will explore this pattern more
in the next chapter.

\textbf{Exercise:} Write a loop that counts how many words appear 200
times. What are they? How many words appear 100 times, 50 times, and 20
times?

\textbf{Optional:} If you know how to define a function, write a
function that takes a \passthrough{\lstinline!Counter!} and a frequency
as arguments, prints all words with that frequency, and returns the
number of words with that frequency.

\section{Summary}\label{summary}

This chapter introduces dictionaries, which are collections of keys and
corresponding values. We used a dictionary to count the number of unique
words in a file and the number of times each one appears.

It also introduces the bracket operator, which selects an element from a
list or tuple, or looks up a key in a dictionary and finds the
corresponding value.

We saw some new methods for working with strings, including
\passthrough{\lstinline!lower!} and \passthrough{\lstinline!strip!}.
Also, we used the \passthrough{\lstinline!unicodedata!} library to
identify characters that are considered punctuation.
