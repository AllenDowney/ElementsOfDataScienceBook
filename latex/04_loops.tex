\chapter{Loops and Files}\label{loops-and-files}

This chapter presents loops, which are used to perform repeated
computation, and files, which are used to store data. As an example, we
will download the famous book \emph{War and Peace} and write a loop that
reads the book and counts the words. This example presents some new
computational tools -- it is also an introduction to working with
textual data.

\section{Loops}\label{loops}

One of the most important elements of computation is repetition, and the
most common way to perform repetitive computations is a
\passthrough{\lstinline!for!} loop. As a simple example, suppose we want
to display the elements of a tuple. Here's a tuple of three integers:
\index{loop}

\begin{lstlisting}[language=Python,style=source]
t = (1, 2, 3)
\end{lstlisting}

And here's a \passthrough{\lstinline!for!} loop that prints the
elements.
\index{element}

\begin{lstlisting}[language=Python,style=source]
for x in t:
    print(x)
\end{lstlisting}

\begin{lstlisting}[style=output]
1
2
3
\end{lstlisting}

The first line of the loop is a \textbf{header} that specifies the
tuple, \passthrough{\lstinline!t!}, and a variable name,
\passthrough{\lstinline!x!}. The tuple must already exists, but if
\passthrough{\lstinline!x!} does not, the loop will create it. Note that
the header ends with a colon, \passthrough{\lstinline!:!}.
\index{header}
\index{colon}

\pagebreak

Inside the loop is a \passthrough{\lstinline!print!} statement, which
displays the value of \passthrough{\lstinline!x!}. So here's what
happens:
\index{print statement}

\begin{enumerate}
\def\labelenumi{\arabic{enumi}.}
\item
  When the loop starts, it gets the first element of
  \passthrough{\lstinline!t!}, which is \passthrough{\lstinline!1!}, and
  assigns it to \passthrough{\lstinline!x!}. It executes the
  \passthrough{\lstinline!print!} statement, which displays the value
  \passthrough{\lstinline!1!}.
\item
  Then it gets the second element of \passthrough{\lstinline!t!}, which
  is \passthrough{\lstinline!2!}, and displays it.
\item
  Then it gets the third element of \passthrough{\lstinline!t!}, which
  is \passthrough{\lstinline!3!}, and displays it.
\end{enumerate}

After printing the last element of the tuple, the loop ends. We can also
loop through the letters in a string:

\begin{lstlisting}[language=Python,style=source]
word = 'Data'

for letter in word:
    print(letter)
\end{lstlisting}

\begin{lstlisting}[style=output]
D
a
t
a
\end{lstlisting}

When the loop begins, \passthrough{\lstinline!word!} already exists, but
\passthrough{\lstinline!letter!} does not. Again, the loop creates
\passthrough{\lstinline!letter!} and assigns values to it. The variable
created by the loop is called the \textbf{loop variable}. You can give
it any name you like -- in this example, I chose
\passthrough{\lstinline!letter!} to remind me what kind of value it
contains. After the loop ends, the loop variable contains the last
value.
\index{loop variable}
\index{variable, loop}

\begin{lstlisting}[language=Python,style=source]
letter
\end{lstlisting}

\begin{lstlisting}[style=output]
'a'
\end{lstlisting}

\textbf{Exercise:} Create a list called
\passthrough{\lstinline!sequence!} with four elements of any type. Write
a \passthrough{\lstinline!for!} loop that prints the elements. Call the
loop variable \passthrough{\lstinline!element!}.

\section{Counting with Loops}\label{counting-with-loops}

Inside a loop, it is common to use a variable to count the number of
times something happens. We've already seen that you can create a
variable and give it a value, like this:

\begin{lstlisting}[language=Python,style=source]
count = 0
count
\end{lstlisting}

\begin{lstlisting}[style=output]
0
\end{lstlisting}

\pagebreak

If you assign a different value to the same variable, the new value
replaces the old one.

\begin{lstlisting}[language=Python,style=source]
count = 1
count
\end{lstlisting}

\begin{lstlisting}[style=output]
1
\end{lstlisting}

You can increase the value of a variable by reading the old value,
adding \passthrough{\lstinline!1!}, and assigning the result back to the
original variable.

\begin{lstlisting}[language=Python,style=source]
count = count + 1
count
\end{lstlisting}

\begin{lstlisting}[style=output]
2
\end{lstlisting}

Increasing the value of a variable is called \textbf{incrementing} and
decreasing the value is called \textbf{decrementing}. These operations
are so common that there are special operators for them.
\index{increment}
\index{decrement}

\begin{lstlisting}[language=Python,style=source]
count += 1
count
\end{lstlisting}

\begin{lstlisting}[style=output]
3
\end{lstlisting}

In this example, the \passthrough{\lstinline!+=!} operator reads the
value of \passthrough{\lstinline!count!}, adds
\passthrough{\lstinline!1!}, and assigns the result back to
\passthrough{\lstinline!count!}. Python also provides
\passthrough{\lstinline!-=!} and other update operators like
\passthrough{\lstinline!*=!} and \passthrough{\lstinline!/=!}.
\index{operator, update}
\index{update operator}

\textbf{Exercise:} The following is a number trick from the website
\emph{Learn With Math Games}:

\begin{quote}
\emph{Finding Someone's Age}

\begin{itemize}
\item
  Ask the person to multiply the first number of their age by 5.
\item
  Tell them to add 3.
\item
  Now tell them to double this figure.
\item
  Finally, have the person add the second number of their age to the
  figure and have them tell you the answer.
\item
  Deduct 6 and you will have their age.
\end{itemize}
\end{quote}

Test this algorithm using your age. Use a single variable and update it
using \passthrough{\lstinline!+=!} and other update operators.

\section{Files}\label{files}

Now that we know how to count, let's see how to read words from a file.
As an example, we'll read a file that contains the text of Tolstoy's
famous novel, \emph{War and Peace}. We can download it from Project
Gutenberg, which is a repository of free books. Instructions are in the
notebook for this chapter.
\index{file}
\index{Project Gutenberg}
\index{War and Peace@\textit{War and Peace}}

In order to read the contents of the file, you have to \textbf{open} it,
which you can do with the \passthrough{\lstinline!open!} function.

\begin{lstlisting}[language=Python,style=source]
fp = open('2600-0.txt')
fp
\end{lstlisting}

\begin{lstlisting}[style=output]
<_io.TextIOWrapper name='2600-0.txt' mode='r' encoding='UTF-8'>
\end{lstlisting}

The result is a \passthrough{\lstinline!TextIOWrapper!}, which is a type
of \textbf{file pointer}. It contains the name of the file, the mode
(which is \passthrough{\lstinline!r!} for ``reading'') and the encoding
(which is \passthrough{\lstinline!UTF!} for ``Unicode Transformation
Format''). A file pointer is like a bookmark -- it keeps track of which
parts of the file you have read.
\index{file pointer}
\index{Unicode}

If you use a file pointer in a \passthrough{\lstinline!for!} loop, it
loops through the lines in the file. So we can count the number of lines
like this:

\begin{lstlisting}[language=Python,style=source]
fp = open('2600-0.txt')
count = 0
for line in fp:
    count += 1
\end{lstlisting}

And then display the result.

\begin{lstlisting}[language=Python,style=source]
count
\end{lstlisting}

\begin{lstlisting}[style=output]
66050
\end{lstlisting}

There are about 66,000 lines in this file.

\section{if Statements}\label{if-statements}

\passthrough{\lstinline!if!} statements are used to check whether a
condition is true and, depending on the result, perform different
computations. A condition is an expression whose value is either
\passthrough{\lstinline!True!} or \passthrough{\lstinline!False!}. For
example, the following expression compares the final value of
\passthrough{\lstinline!count!} to a number:
\index{if statement}
\index{condition}

\begin{lstlisting}[language=Python,style=source]
count > 60000
\end{lstlisting}

\begin{lstlisting}[style=output]
True
\end{lstlisting}

For \emph{War and Peace}, the result is \passthrough{\lstinline!True!}.

\pagebreak

We can use this condition in an \passthrough{\lstinline!if!} statement
to display a message, or not, depending on the result.

\begin{lstlisting}[language=Python,style=source]
if count > 60000:
    print('Long book!')
\end{lstlisting}

\begin{lstlisting}[style=output]
Long book!
\end{lstlisting}

The first line specifies the condition we're checking for. Like the
header of a \passthrough{\lstinline!for!} statement, the first line of
an \passthrough{\lstinline!if!} statement has to end with a colon.
\index{colon}

If the condition is true, the indented statement runs; otherwise, it
doesn't. In the previous example, the condition is true, so the
\passthrough{\lstinline!print!} statement runs. In the following
example, the condition is false, so the \passthrough{\lstinline!print!}
statement doesn't run.
\index{indentation}

\begin{lstlisting}[language=Python,style=source]
if count < 1000:
    print('Short book!')
\end{lstlisting}

We can put an \passthrough{\lstinline!if!} statement inside a
\passthrough{\lstinline!for!} loop. The following example only prints a
line from the book when \passthrough{\lstinline!count!} is
\passthrough{\lstinline!0!}. The other lines are read, but not
displayed.

\begin{lstlisting}[language=Python,style=source]
fp = open('2600-0.txt')
count = 0
for line in fp:
    if count == 0:
        print(line)
    count += 1
\end{lstlisting}

\begin{lstlisting}[style=output]
The Project Gutenberg EBook of War and Peace, by Leo Tolstoy
\end{lstlisting}

Notice that we use \passthrough{\lstinline!==!} to compare values and
check if they are equal, not \passthrough{\lstinline!=!}, which is used
in assignment statements. Also, notice the indentation in this example:

\begin{itemize}
\item
  Statements inside the \passthrough{\lstinline!for!} loop are indented.
\item
  The statement inside the \passthrough{\lstinline!if!} statement is
  indented.
\item
  The statement \passthrough{\lstinline!count += 1!} is
  \textbf{outdented} from the previous line, so it ends the
  \passthrough{\lstinline!if!} statement. But it is still inside the
  \passthrough{\lstinline!for!} loop.
\end{itemize}

It is legal in Python to use spaces or tabs for indentation, but the
most common convention is to use four spaces, never tabs.
\index{indentation}

\section{\texorpdfstring{The \texttt{break}
Statement}{The break Statement}}\label{the-break-statement}

If we display the final value of \passthrough{\lstinline!count!}, we see
that the loop reads the entire file, but only prints one line:

\begin{lstlisting}[language=Python,style=source]
count
\end{lstlisting}

\begin{lstlisting}[style=output]
66050
\end{lstlisting}

We can avoid reading the whole file by using a
\passthrough{\lstinline!break!} statement, like this:
\index{break statement}

\begin{lstlisting}[language=Python,style=source]
fp = open('2600-0.txt')
count = 0
for line in fp:
    print(line)
    count += 1
    if count == 1:
        break
\end{lstlisting}

\begin{lstlisting}[style=output]
The Project Gutenberg EBook of War and Peace, by Leo Tolstoy
\end{lstlisting}

The \passthrough{\lstinline!break!} statement ends the loop immediately,
skipping the rest of the file, as we can confirm by checking the final
value of \passthrough{\lstinline!count!}.

\begin{lstlisting}[language=Python,style=source]
count
\end{lstlisting}

\begin{lstlisting}[style=output]
1
\end{lstlisting}

\textbf{Exercise:} Write a loop that prints the first 5 lines of the
file and then breaks out of the loop.

\section{Whitespace}\label{whitespace}

If we run the loop again and display the final value of
\passthrough{\lstinline!line!}, we see the special sequence
\passthrough{\lstinline!\\n!} at the end.
\index{whitespace}

\begin{lstlisting}[language=Python,style=source]
fp = open('2600-0.txt')
count = 0
for line in fp:
    count += 1
    if count == 1:
        break

line
\end{lstlisting}

\begin{lstlisting}[style=output]
'The Project Gutenberg EBook of War and Peace, by Leo Tolstoy\n'
\end{lstlisting}

This sequence represents a single character, called a \textbf{newline},
that puts vertical space between lines. If we use a
\passthrough{\lstinline!print!} statement to display
\passthrough{\lstinline!line!}, we don't see the special sequence, but
we do see extra space after the line.
\index{newline}

\begin{lstlisting}[language=Python,style=source]
print(line)
\end{lstlisting}

\begin{lstlisting}[style=output]
The Project Gutenberg EBook of War and Peace, by Leo Tolstoy

\end{lstlisting}

In other strings, you might see the sequence
\passthrough{\lstinline!\\t!}, which represents a tab character. When
you print a tab character, it adds enough space to make the next
character appear in a column that is a multiple of 8.
\index{tab character}

\begin{lstlisting}[language=Python,style=source]
print('|       ' * 6)
print('a\tbc\tdef\tghij\tklmno\tpqrstu')
\end{lstlisting}

\begin{lstlisting}[style=output]
|       |       |       |       |       |
a       bc      def     ghij    klmno   pqrstu
\end{lstlisting}

Newline characters, tabs, and spaces are called \textbf{whitespace}
because when they are printed they leave white space on the page
(assuming that the background color is white).

\section{Counting Words}\label{counting-words}

So far we've counted the lines in a file -- now let's count the words.
To split a line into words, we can use a function called
\passthrough{\lstinline!split!} that takes a string and returns a list
of words. To be more precise, \passthrough{\lstinline!split!} doesn't
actually know what a word is -- it just splits the line wherever there's
a space or other whitespace character.
\index{split method}

\begin{lstlisting}[language=Python,style=source]
line.split()
\end{lstlisting}

\begin{lstlisting}[style=output]
['The',
 'Project',
 'Gutenberg',
 'EBook',
 'of',
 'War',
 'and',
 'Peace,',
 'by',
 'Leo',
 'Tolstoy']
\end{lstlisting}

Notice that the syntax for \passthrough{\lstinline!split!} is different
from other functions we have seen. Normally when we call a function, we
name the function and provide values in parentheses. So you might have
expected to write \passthrough{\lstinline!split(line)!}.

\pagebreak

Sadly, that doesn't work.

\begin{lstlisting}[language=Python,style=source]
%%expect NameError

split(line)
\end{lstlisting}

\begin{lstlisting}[style=output]
NameError: name 'split' is not defined
\end{lstlisting}

The problem is that the \passthrough{\lstinline!split!} function belongs
to the string \passthrough{\lstinline!line!}. In a sense, the function
is attached to the string, so we can only refer to it using the string
and the \textbf{dot operator}, which is the period between
\passthrough{\lstinline!line!} and \passthrough{\lstinline!split!}. For
historical reasons, functions like this are called \textbf{methods}.
\index{method}

Now that we can split a line into a list of words, we can use
\passthrough{\lstinline!len!} to get the number of words in each list,
and increment \passthrough{\lstinline!count!} accordingly.

\begin{lstlisting}[language=Python,style=source]
fp = open('2600-0.txt')
count = 0
for line in fp:
    count += len(line.split())

count
\end{lstlisting}

\begin{lstlisting}[style=output]
566316
\end{lstlisting}

By this count, there are more than half a million words in \emph{War and
Peace}.

Actually, there aren't quite that many, because the file we got from
Project Gutenberg has some introductory material before the text and
some license information at the end. To mark the beginning and end of
the text, the file includes special lines that begin with
\passthrough{\lstinline!'***'!}. We can identify these lines with the
\passthrough{\lstinline!startswith!} function, which checks whether a
string begins with a particular sequence of characters.
\index{startswith (string method)}

\begin{lstlisting}[language=Python,style=source]
line = '*** START OF THIS PROJECT GUTENBERG EBOOK WAR AND PEACE ***'
line.startswith('***')
\end{lstlisting}

\begin{lstlisting}[style=output]
True
\end{lstlisting}

To skip the front matter, we can use a loop to read lines until it finds
the first line that starts with this sequence.

\pagebreak

Then we can use a second loop to read lines and count words until it
finds the second line that starts with this sequence.

\begin{lstlisting}[language=Python,style=source]
fp = open('2600-0.txt')
for line in fp:
    if line.startswith('***'):
        print(line)
        break

count = 0
for line in fp:
    if line.startswith('***'):
        print(line)
        break
    count += len(line.split())
\end{lstlisting}

\begin{lstlisting}[style=output]
*** START OF THIS PROJECT GUTENBERG EBOOK WAR AND PEACE ***

*** END OF THIS PROJECT GUTENBERG EBOOK WAR AND PEACE ***

\end{lstlisting}

When the second loop exits, \passthrough{\lstinline!count!} contains the
number of words in the text.

\begin{lstlisting}[language=Python,style=source]
count
\end{lstlisting}

\begin{lstlisting}[style=output]
563299
\end{lstlisting}

\section{Summary}\label{summary}

This chapter presents loops, \passthrough{\lstinline!if!} statements,
and the \passthrough{\lstinline!break!} statement. It also introduces
tools for working with letters and words, and a simple kind of textual
analysis, word counting.

In the next chapter we'll continue this example, counting the number of
unique words in a text and the number of times each word appears. And
we'll see another way to represent a collection of values, a Python
dictionary.
