\hypertarget{elements-of-data-science}{%
\section{Elements of Data Science}\label{elements-of-data-science}}

\emph{Elements of Data Science} is an introduction to data science for
people with no programming experience. My goal is to present a small,
powerful subset of Python that allows you to do real work in data
science as quickly as possible.

I don't assume that the reader knows anything about programming,
statistics, or data science. When I use a term, I try to define it
immediately, and when I use a programming feature, I try to explain it.

This book is in the form of Jupyter notebooks. Jupyter is a software
development tool you can run in a web browser, so you don't have to
install any software. A Jupyter notebook is a document that contains
text, Python code, and results. So you can read it like a book, but you
can also modify the code, run it, develop new programs, and test them.

The notebooks contain exercises where you can practice what you learn.
Most of the exercises are meant to be quick, but a few are more
substantial.

The license for this book is the
\href{https://creativecommons.org/licenses/by-nc-sa/4.0/}{Creative
Commons Attribution-NonCommercial-ShareAlike 4.0 International (CC
BY-NC-SA 4.0)}.

This material is a work in progress, so suggestions are welcome. The
best way to provide feedback is to
\href{https://github.com/AllenDowney/ElementsOfDataScience/issues}{click
here and create an issue in this GitHub repository}.

\hypertarget{case-studies}{%
\subsection{Case Studies}\label{case-studies}}

In addition to the notebooks below, the \emph{Elements of Data Science}
curriculum includes these case studies:

\begin{itemize}
\item
  \href{https://allendowney.github.io/PoliticalAlignmentCaseStudy/}{Political
  Alignment Case Study}: Using data from the General Social Survey, this
  case study explore changing opinions on a variety of topics among
  survey respondents in the United States. Readers choose one of about
  120 survey questions and see how responses have changed over time and
  how these changes relate to political alignment (conservative,
  moderate, or liberal).
\item
  \href{https://allendowney.github.io/RecidivismCaseStudy/}{Recidivism
  Case Study}: This case study is based on a well known paper, ``Machine
  Bias'', which was published by Politico in 2016. It relates to COMPAS,
  a statistical tool used in the criminal justice system to assess the
  risk that a defendant will commit another crime if released. The
  ProPublica article concludes that COMPAS is unfair to Black defendants
  because they are more likely to be misclassified as high risk. A
  response article in the Washington Post suggests that ``It's actually
  not that clear.'' Using the data from the original article, this case
  study explains the (many) metrics used to evaluate binary classifiers,
  shows the challenges of defining algorithmic fairness, and starts a
  discussion of the context, ethics, and social impact of data science.
\item
  \href{https://allendowney.github.io/BiteSizeBayes/}{Bite Size Bayes}:
  An introduction to probability with a focus on Bayes's Theorem.
\item
  \href{https://allendowney.github.io/AstronomicalData/}{Astronomical
  Data in Python}: An introduction to SQL using data from the Gaia space
  telescope as an example.
\end{itemize}

\hypertarget{the-notebooks}{%
\subsection{The notebooks}\label{the-notebooks}}

For each of the notebooks below, you have three options:

\begin{itemize}
\item
  If you view the notebook on NBViewer, you can read it, but you can't
  run the code.
\item
  If you run the notebook on Colab, you'll be able to run the code, do
  the exercises, and save your modified version of the notebook in a
  Google Drive (if you have one).
\item
  Or, if you download the notebook, you can run it in your own
  environment. But in that case it is up to you to make sure you have
  the libraries you need.
\end{itemize}

\hypertarget{notebook-1}{%
\subsubsection{Notebook 1}\label{notebook-1}}

\textbf{Variables and values}: The first notebook explains how to use
Jupyter and introduces variables, values, and numerical computation.

\href{https://colab.research.google.com/github/AllenDowney/ElementsOfDataScience/blob/master/01_variables.ipynb}{Click
here to run this notebook on Colab}

\href{https://github.com/AllenDowney/ElementsOfDataScience/raw/master/01_variables.ipynb}{or
click here to download it}

\hypertarget{notebook-2}{%
\subsubsection{Notebook 2}\label{notebook-2}}

\textbf{Times and places}: This notebook shows how to represent times,
dates, and locations in Python, and uses the GeoPandas library to plot
points on a map.

\href{https://colab.research.google.com/github/AllenDowney/ElementsOfDataScience/blob/master/02_times.ipynb}{Click
here to run this notebook on Colab}

\href{https://github.com/AllenDowney/ElementsOfDataScience/raw/master/02_times.ipynb}{or
click here to download it}

\hypertarget{notebook-3}{%
\subsubsection{Notebook 3}\label{notebook-3}}

\textbf{Lists and Arrays}: This notebook presents lists and NumPy
arrays. It discusses absolute, relative, and percent errors, and ways to
summarize them.

\href{https://colab.research.google.com/github/AllenDowney/ElementsOfDataScience/blob/master/03_arrays.ipynb}{Click
here to run this notebook on Colab}

\href{https://github.com/AllenDowney/ElementsOfDataScience/raw/master/03_arrays.ipynb}{or
click here to download it}

\hypertarget{notebook-4}{%
\subsubsection{Notebook 4}\label{notebook-4}}

\textbf{Loops and Files}: This notebook presents the \texttt{for} loop
and the \texttt{if} statement; then it uses them to speed-read \emph{War
and Peace} and count the words.

\href{https://colab.research.google.com/github/AllenDowney/ElementsOfDataScience/blob/master/04_loops.ipynb}{Click
here to run this notebook on Colab}

\href{https://github.com/AllenDowney/ElementsOfDataScience/raw/master/04_loops.ipynb}{or
click here to download it}

\hypertarget{notebook-5}{%
\subsubsection{Notebook 5}\label{notebook-5}}

\textbf{Dictionaries}: This notebook presents one of the most powerful
features of Python, dictionaries, and uses them to count the unique
words in a text and their frequencies.

\href{https://colab.research.google.com/github/AllenDowney/ElementsOfDataScience/blob/master/05_dictionaries.ipynb}{Click
here to run this notebook on Colab}

\href{https://github.com/AllenDowney/ElementsOfDataScience/raw/master/05_dictionaries.ipynb}{or
click here to download it}

\hypertarget{notebook-6}{%
\subsubsection{Notebook 6}\label{notebook-6}}

\textbf{Plotting}: This notebook introduces a plotting library,
Matplotlib, and uses it to generate a few common data visualizations and
one less common one, a Zipf plot.

\href{https://colab.research.google.com/github/AllenDowney/ElementsOfDataScience/blob/master/06_plotting.ipynb}{Click
here to run this notebook on Colab}

\href{https://github.com/AllenDowney/ElementsOfDataScience/raw/master/06_plotting.ipynb}{or
click here to download it}

\hypertarget{notebook-7}{%
\subsubsection{Notebook 7}\label{notebook-7}}

\textbf{DataFrames}: This notebook presents DataFrames, which are used
to represent tables of data. As an example, it uses data from the
National Survey of Family Growth to find the average weight of babies in
the U.S.

\href{https://colab.research.google.com/github/AllenDowney/ElementsOfDataScience/blob/master/07_dataframes.ipynb}{Click
here to run this notebook on Colab}

\href{https://github.com/AllenDowney/ElementsOfDataScience/raw/master/07_dataframes.ipynb}{or
click here to download it}

\hypertarget{notebook-8}{%
\subsubsection{Notebook 8}\label{notebook-8}}

\textbf{Distributions}: This notebook explains what a distribution is
and presents 3 ways to represent one: a PMF, CDF, or PDF. It also shows
how to compare a distribution to another distribution or a mathematical
model.

\href{https://colab.research.google.com/github/AllenDowney/ElementsOfDataScience/blob/master/08_distributions.ipynb}{Click
here to run this notebook on Colab}

\href{https://github.com/AllenDowney/ElementsOfDataScience/raw/master/08_distributions.ipynb}{or
click here to download it}

\hypertarget{notebook-9}{%
\subsubsection{Notebook 9}\label{notebook-9}}

\textbf{Relationships}: This notebook explores relationships between
variables using scatter plots, violin plots, and box plots. It
quantifies the strength of a relationship using the correlation
coefficient and uses simple regression to estimate the slope of a line.

\href{https://colab.research.google.com/github/AllenDowney/ElementsOfDataScience/blob/master/09_relationships.ipynb}{Click
here to run this notebook on Colab}

\href{https://github.com/AllenDowney/ElementsOfDataScience/raw/master/09_relationships.ipynb}{or
click here to download it}

\hypertarget{notebook-10}{%
\subsubsection{Notebook 10}\label{notebook-10}}

\textbf{Regression}: This notebook presents multiple regression and uses
it to explore the relationship between age, education, and income. It
uses visualization to interpret multivariate models. It also presents
binary variables and logistic regression.

\href{https://colab.research.google.com/github/AllenDowney/ElementsOfDataScience/blob/master/10_regression.ipynb}{Click
here to run this notebook on Colab}

\href{https://github.com/AllenDowney/ElementsOfDataScience/raw/master/10_regression.ipynb}{or
click here to download it}

\hypertarget{notebook-11}{%
\subsubsection{Notebook 11}\label{notebook-11}}

\textbf{Resampling}: This notebook presents computational methods we can
use to quantify variation due to random sampling, which is one of
several sources of error in statistical estimation.

\href{https://colab.research.google.com/github/AllenDowney/ElementsOfDataScience/blob/master/11_resampling.ipynb}{Click
here to run this notebook on Colab}

\href{https://github.com/AllenDowney/ElementsOfDataScience/raw/master/11_resampling.ipynb}{or
click here to download it}

\hypertarget{notebook-12}{%
\subsubsection{Notebook 12}\label{notebook-12}}

\textbf{Bootstrapping}: Bootstrapping is a kind of resampling that is
well suited to the kind of survey data we've been working with.

\href{https://colab.research.google.com/github/AllenDowney/ElementsOfDataScience/blob/master/12_bootstrap.ipynb}{Click
here to run this notebook on Colab}

\href{https://github.com/AllenDowney/ElementsOfDataScience/raw/master/12_bootstrap.ipynb}{or
click here to download it}

\hypertarget{notebook-13}{%
\subsubsection{Notebook 13}\label{notebook-13}}

\textbf{Hypothesis Testing}: Hypothesis testing is the bugbear of
classical statistics. This notebook presents a computational approach to
the topic that makes it clear that
\href{http://allendowney.blogspot.com/2016/06/there-is-still-only-one-test.html}{there
is only one test}.

\href{https://colab.research.google.com/github/AllenDowney/ElementsOfDataScience/blob/master/13_hypothesis.ipynb}{Click
here to run this notebook on Colab}

\href{https://github.com/AllenDowney/ElementsOfDataScience/raw/master/13_hypothesis.ipynb}{or
click here to download it}
