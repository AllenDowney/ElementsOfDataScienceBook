\hypertarget{political-alignment-case-study}{%
\section{Political Alignment Case
Study}\label{political-alignment-case-study}}

This data science case study explores the relationship between political
alignment (conservative, moderate, or liberal) and other attitudes and
beliefs. It is meant primarily for teaching and learning about data
science, but we might do some research along the way.

If you are working through
\href{https://allendowney.github.io/ElementsOfDataScience/}{the
\emph{Elements of Data Science} curriculum}, you should be ready to
start this case study when you have completed Notebook 6, which covers
basic Pandas.

This material is a work in progress, so your feedback is welcome. The
best way to provide that feedback is to
\href{https://github.com/AllenDowney/PoliticalAlignmentCaseStudy/issues}{click
here and create an issue in this GitHub repository}.

Uodate August 2022: I have updated the notebooks and data files with the
2021 data from the General Social Survey.

\hypertarget{videos-and-slides}{%
\subsection{Videos and Slides}\label{videos-and-slides}}

I presented this case study for
\href{https://global.pydata.org/talks/363}{PyData Global 2020}.

Here are the
\href{https://docs.google.com/presentation/d/e/2PACX-1vSqifcdVGQmMoLDNlmbnugZ58jieItA_weGEF9oRsQCAa6iICLmehevGRzINYVv0tCGqcSTvuIQOSJo/pub}{slides
I presented}

And here are the videos:

\begin{itemize}
\item
  \href{https://www.youtube.com/watch?v=6Bg7v5EGiWY}{Part 1}
\item
  \href{https://www.youtube.com/watch?v=y8A3bKjpJe4}{Part 2}
\item
  \href{https://www.youtube.com/watch?v=vfuXyXXSNtM}{Part 3}
\item
  \href{https://www.youtube.com/watch?v=XFPi-AUiJSo}{Part 4}
\end{itemize}

\hypertarget{the-notebooks}{%
\subsection{The notebooks}\label{the-notebooks}}

For each of the notebooks below, you have two options: if you view the
notebook on NBViewer, you can read it, but you can't run the code. If
you run the notebook on Colab, you'll be able to run the code, do the
exercises, and save your modified version of the notebook in a Google
Drive (if you have one).

\hypertarget{notebook-1}{%
\subsubsection{Notebook 1}\label{notebook-1}}

\textbf{Cleaning and validation}: The first notebook loads data from the
General Social Survey (GSS) and walks through the process of cleaning
and validating the data. At the end, you can help me by choosing a
random variable, checking the values against the codebook, and reporting
your results.

\begin{itemize}
\item
  \href{https://nbviewer.jupyter.org/github/AllenDowney/PoliticalAlignmentCaseStudy/blob/master/01_clean.ipynb}{Click
  here to read Notebook 1 on NBViewer}
\item
  \href{https://colab.research.google.com/github/AllenDowney/PoliticalAlignmentCaseStudy/blob/master/01_clean.ipynb}{Click
  here to run Notebook 1 on Colab}
\end{itemize}

\hypertarget{notebook-2}{%
\subsubsection{Notebook 2}\label{notebook-2}}

\textbf{Exploration}: This notebook uses the tools of exploratory data
analysis to look at survey responses about political alignment. It uses
PMFs to display distributions, time series to represent changes over
time, and cross tabulation to look at changes in distribution over time.
It also introduces local regression as a way to plot a smooth line
through noisy data.

\begin{itemize}
\item
  \href{https://nbviewer.jupyter.org/github/AllenDowney/PoliticalAlignmentCaseStudy/blob/master/02_polviews.ipynb}{Click
  here to read Notebook 2 on NBViewer}
\item
  \href{https://colab.research.google.com/github/AllenDowney/PoliticalAlignmentCaseStudy/blob/master/02_polviews.ipynb}{Click
  here to run Notebook 2 on Colab}
\end{itemize}

\hypertarget{notebook-3}{%
\subsubsection{Notebook 3}\label{notebook-3}}

\textbf{Political alignment and outlook}: This notebook explores the
relationship between political alignment and three survey questions
related to ``outlook''. It uses a pivot table to compute the mean of the
response variable grouped by political alignment and time.

\begin{itemize}
\item
  \href{https://nbviewer.jupyter.org/github/AllenDowney/PoliticalAlignmentCaseStudy/blob/master/03_outlook.ipynb}{Click
  here to read Notebook 3 on NBViewer}
\item
  \href{https://colab.research.google.com/github/AllenDowney/PoliticalAlignmentCaseStudy/blob/master/03_outlook.ipynb}{Click
  here to run Notebook 3 on Colab}
\end{itemize}

\hypertarget{notebook-4}{%
\subsubsection{Notebook 4}\label{notebook-4}}

\textbf{Political alignment and other beliefs}: This notebook explores
the relationship between political alignment and other attitudes and
beliefs. It is a template for a do-it-yourself,
choose-your-own-adventure mini-project, where you have the chance to
explore a variable in the GSS dataset and report the results.

\begin{itemize}
\item
  \href{https://nbviewer.jupyter.org/github/AllenDowney/PoliticalAlignmentCaseStudy/blob/master/04_worldview.ipynb}{Click
  here to read Notebook 4 on NBViewer}
\item
  \href{https://colab.research.google.com/github/AllenDowney/PoliticalAlignmentCaseStudy/blob/master/04_worldview.ipynb}{Click
  here to run Notebook 4 on Colab}
\end{itemize}

Copyright 2020 Allen B. Downey

License:
\href{https://creativecommons.org/licenses/by-nc-sa/4.0/}{Attribution-NonCommercial-ShareAlike
4.0 International (CC BY-NC-SA 4.0)}
