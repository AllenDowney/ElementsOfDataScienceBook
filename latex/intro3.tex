\hypertarget{recidivism-case-study}{%
\section{Recidivism Case Study}\label{recidivism-case-study}}

This case study is about classification and algorithmic fairness. It is
based on two articles that were published in 2016:

\begin{itemize}
\item
  ``\href{https://www.propublica.org/article/machine-bias-risk-assessments-in-criminal-sentencing}{Machine
  Bias}'', by Julia Angwin, Jeff Larson, Surya Mattu and Lauren
  Kirchner, and published by
  \href{https://www.propublica.org}{ProPublica}.
\item
  A response by Sam Corbett-Davies, Emma Pierson, Avi Feller and Sharad
  Goel:
  ``\href{https://www.washingtonpost.com/news/monkey-cage/wp/2016/10/17/can-an-algorithm-be-racist-our-analysis-is-more-cautious-than-propublicas/}{A
  computer program used for bail and sentencing decisions was labeled
  biased against blacks. It's actually not that clear.}'', published in
  the Washington Post.
\end{itemize}

Both articles are about
\href{https://en.wikipedia.org/wiki/COMPAS_(software)}{COMPAS}, a
statistical tool used in the justice system to assign defendants a
``risk score'' that is intended to reflect the risk that they will
commit another crime if released.

The ProPublica article evaluates COMPAS as a binary classifier and
compares its error rates for black and white defendants. It concludes
that COMPAS is unfair to black defendants because they are more likely
to be misclassified as high risk.

In response, the Washington Post article shows that COMPAS has the same
predictive value for black and white defendants. And they explain that
the test cannot have the same predictive value and the same error rates
at the same time.

The purpose of this case study is to understand these conflicting
claims, to learn about classification algorithms and the metrics we use
to evaluate them, and to think about fairness and the ethics of data
science.

If you are working through
\href{https://allendowney.github.io/ElementsOfDataScience}{the
\emph{Elements of Data Science} curriculum}, you should be ready to
start this case study when you have completed Notebook 6, which covers
basic Pandas, and the
\href{https://allendowney.github.io/PoliticalAlignmentCaseStudy}{Political
Alignment Case Study}, which introduces cross-tabulation and some of the
other tools we'll use.

\hypertarget{the-notebooks}{%
\subsection{The notebooks}\label{the-notebooks}}

\begin{itemize}
\item
  \href{https://colab.research.google.com/github/AllenDowney/RecidivismCaseStudy/blob/master/01_classification.ipynb}{In
  the first notebook} I replicate the analysis from the ProPublica
  article and define the basic metrics we use to evaluate classification
  algorithms, including error rates and predictive values.
\item
  \href{https://colab.research.google.com/github/AllenDowney/RecidivismCaseStudy/blob/master/02_calibration.ipynb}{In
  the second notebook} I replicate the analysis from the WaPo article
  and define the calibration curve, the ROC curve, and a related metric,
  concordance.
\item
  \href{https://colab.research.google.com/github/AllenDowney/RecidivismCaseStudy/blob/master/03_fairness.ipynb}{In
  the third notebook} I use the same methods to evaluate the performance
  of COMPAS for male and female defendants, and lay out the fundamental
  conflict between two definitions of fairness.
\end{itemize}

These three notebooks are intended to support a module in a data science
class that engages students in the context and ethical challenges of
machine learning.

\hypertarget{slides}{%
\subsection{Slides}\label{slides}}

I used these notebooks for a module of my
\href{https://sites.google.com/site/olinds20/}{Data Science class at
Olin College}.

Over the course of three class sessions, I
\href{https://github.com/AllenDowney/RecidivismCaseStudy/raw/master/Recidivism\%20Case\%20Study.pdf}{presented
these slides} and led a discussion with students.

\hypertarget{additional-notebooks}{%
\subsection{Additional notebooks}\label{additional-notebooks}}

This repository contains three additional notebooks with additional
explorations that you might be interested in. They are not essential to
understand the issues, and they are less complete than the first three
notebooks.

\begin{itemize}
\item
  \href{https://colab.research.google.com/github/AllenDowney/RecidivismCaseStudy/blob/master/04_matrix.ipynb}{The
  fourth notebook} proves what I asserted in the second notebook: if you
  are given prevalence and error rates, you can compute predictive
  values; and if you are given prevalence and predictive values, you can
  compute error rates.
\item
  \href{https://colab.research.google.com/github/AllenDowney/RecidivismCaseStudy/blob/master/05_subgroups.ipynb}{The
  fifth notebook} demonstrates that the challenge of defining fairness
  between groups gets harder as we consider more groups, and identifies
  the groups with the highest and lowest errors and predictive values.
\item
  \href{https://colab.research.google.com/github/AllenDowney/RecidivismCaseStudy/blob/master/06_error.ipynb}{The
  sixth notebook} explores what I call ``the other calibration curve'',
  the probability of being classified high risk as a function of the
  probability of recidivism.
\end{itemize}

I include these notebook in part to resist the temptation to hide my
development process. I worked on this case study on and off over several
years. I explored a lot of things and took a lot of wrong turns. It took
me a long time to find the story, get it organized, and strike a balance
between two conflicting goals: maintaining the scientific detachment
that lets us tackle difficult topics while also keeping sight of the
context, the people, and the human consequences.

I hope these materials will be engaging and informative for readers, and
useful for teaching and learning the ethical practice of data science.
