\chapter{Political Alignment and
Polarization}\label{political-alignment-and-polarization}

This chapter and the next make up a case study that uses data from the
General Social Survey (GSS) to explore political beliefs and political
alignment (conservative, moderate, or liberal) in the United States. In
this chapter, we will:

\begin{enumerate}
\def\labelenumi{\arabic{enumi}.}
\item
  Compare the distributions of political alignment from 1974 and 2022.
\item
  Plot the mean and standard deviation of responses over time as a way
  of quantifying shifts in political alignment and polarization.
\item
  Use local regression to plot a smooth line through noisy data.
\item
  Use cross tabulation to compute the fraction of respondents in each
  category over time.
\item
  Plot the results using a custom color palette.
\end{enumerate}

As an exercise, you will look at changes in political party affiliation
over the same period. In the next chapter, we'll use the same dataset to
explore the relationship between political alignment and other attitudes
and beliefs.

We'll use an extract of the data that I have cleaned and resampled to
correct for stratified sampling. Instructions for downloading the file
are in the notebook for this chapter. It contains three resamplings --
we'll use the first, \passthrough{\lstinline!gss0!}, to get started.

\begin{lstlisting}[language=Python,style=source]
datafile = 'gss_pacs_resampled.hdf'
gss = pd.read_hdf(datafile, key='gss0')
gss.shape
\end{lstlisting}

\begin{lstlisting}[style=output]
(72390, 207)
\end{lstlisting}

\section{Political Alignment}\label{political-alignment}

The people surveyed for the GSS were asked about their ``political
alignment'', which is where they place themselves on a spectrum from
liberal to conservative. They were asked:
\index{General Social Survey}
\index{liberal}
\index{conservative}
\index{moderate}
\index{alignment, political}
\index{political alignment}

\begin{quote}
We hear a lot of talk these days about liberals and conservatives. I'm
going to show you a seven-point scale on which the political views that
people might hold are arranged from extremely liberal--point 1--to
extremely conservative--point 7. Where would you place yourself on this
scale?
\end{quote}

Here is the scale they were shown:

\begin{longtable}[]{ll}
\toprule
Code & Response \\
\midrule
\endhead
\bottomrule
\endlastfoot
1 & Extremely liberal \\
2 & Liberal \\
3 & Slightly liberal \\
4 & Moderate \\
5 & Slightly conservative \\
6 & Conservative \\
7 & Extremely conservative \\
\end{longtable}

The variable \passthrough{\lstinline!polviews!} contains the responses.

\begin{lstlisting}[language=Python,style=source]
polviews = gss['polviews']
\end{lstlisting}

To see how the responses have changed over time, we'll inspect them at
the beginning and end of the observation period. First we'll make a
Boolean Series that's \passthrough{\lstinline!True!} for responses from
1974, and select the responses from 1974.
\index{Boolean Series}

\begin{lstlisting}[language=Python,style=source]
year74 = (gss['year'] == 1974)
polviews74 = polviews[year74]
\end{lstlisting}

And we can do the same for 2022.

\begin{lstlisting}[language=Python,style=source]
year22 = (gss['year'] == 2022)
polviews22 = polviews[year22]
\end{lstlisting}

\pagebreak

We'll use the following function to count the number of times each
response occurs.
\index{values function}

\begin{lstlisting}[language=Python,style=source]
def values(series):
    '''Count the values and sort.

    series: pd.Series

    returns: series mapping from values to frequencies
    '''
    return series.value_counts(dropna=False).sort_index()
\end{lstlisting}

Here are the responses from 1974.

\begin{lstlisting}[language=Python,style=source]
values(polviews74)
\end{lstlisting}

\begin{lstlisting}[style=output]
polviews
1.0     31
2.0    201
3.0    211
4.0    538
5.0    223
6.0    181
7.0     30
NaN     69
Name: count, dtype: int64
\end{lstlisting}

And here are the responses from 2022.

\begin{lstlisting}[language=Python,style=source]
values(polviews22)
\end{lstlisting}

\begin{lstlisting}[style=output]
polviews
1.0     184
2.0     433
3.0     391
4.0    1207
5.0     472
6.0     548
7.0     194
NaN     115
Name: count, dtype: int64
\end{lstlisting}

Looking at these counts, we can get an idea of what the distributions
look like, but in the next section we'll get a clearer picture by
plotting them.

\section{Visualizing Distributions}\label{visualizing-distributions}

To visualize these distributions, we'll use a Probability Mass Function
(PMF), which is similar to a histogram, but there are two differences:
\index{probability mass function}
\index{PMF}


\begin{itemize}
\item
  In a histogram, values are often put in bins, with more than one value
  in each bin. In a PMF each value gets its own bin.
\item
  A histogram computes a count, that is, how many times each value
  appears; a PMF computes a probability, that is, what fraction of the
  time each value appears.
\end{itemize}

We'll use the \passthrough{\lstinline!Pmf!} class from
\passthrough{\lstinline!empiricaldist!} to compute a PMF.
\index{empiricaldist library}
\index{from\_seq (empiricaldist function)}

\begin{lstlisting}[language=Python,style=source]
from empiricaldist import Pmf

pmf74 = Pmf.from_seq(polviews74)
pmf74
\end{lstlisting}

\begin{tabular}{lr}
\toprule
 & probs \\
polviews &  \\
\midrule
1 & 0.0219081 \\
2 & 0.142049 \\
3 & 0.149117 \\
4 & 0.380212 \\
5 & 0.157597 \\
6 & 0.127915 \\
7 & 0.0212014 \\
\bottomrule
\end{tabular}

The following cell imports the function we'll use to decorate the axes
in plots.
\index{decorate function}

\begin{lstlisting}[language=Python,style=source]
from utils import decorate
\end{lstlisting}

Here's the distribution from 1974:

\begin{lstlisting}[language=Python,style=source]
pmf74.bar(label='1974', color='C0', alpha=0.7)

decorate(
    xlabel='Political view on a 7-point scale',
    ylabel='Fraction of respondents',
    title='Distribution of political views',
)
\end{lstlisting}

\begin{center}
\includegraphics[scale=0.6666666]{14_polviews_files/14_polviews_34_0.png}
\end{center}

And from 2022:

\begin{lstlisting}[language=Python,style=source]
pmf22 = Pmf.from_seq(polviews22)
pmf22.bar(label='2022', color='C1', alpha=0.7)

decorate(
    xlabel='Political view on a 7-point scale',
    ylabel='Fraction of respondents',
    title='Distribution of political views',
)
\end{lstlisting}

\begin{center}
\includegraphics[scale=0.6666666]{14_polviews_files/14_polviews_36_0.png}
\end{center}

In both cases, the most common response is \passthrough{\lstinline!4!},
which is the code for `moderate'. Few respondents describe themselves as
`extremely' liberal or conservative. So maybe we're not so polarized
after all.
\index{polarization, political}
\index{political polarization}

\pagebreak

To make it easier to compare the distributions, I'll plot them side by
side.

\begin{lstlisting}[language=Python,style=source]
df = pd.DataFrame({'pmf74': pmf74, 'pmf22': pmf22})
df.plot(kind='bar')

decorate(
    xlabel='Political view on a 7-point scale',
    ylabel='Fraction of respondents',
    title='Distribution of political views',
)
\end{lstlisting}

\begin{center}
\includegraphics[scale=0.6666666]{14_polviews_files/14_polviews_39_0.png}
\end{center}

Now we can see the changes in the distribution more clearly -- the
fraction of people at the extremes (1, 6, and 7) has increased, and the
fraction of people near the middle (2, 3, 4, and 5) has decreased.

\textbf{Exercise:} To summarize these changes, we can compare the mean
and standard deviation of \passthrough{\lstinline!polviews!} in 1974 and
2022. The mean of the responses measures the balance of people in the
population with liberal or conservative leanings. If the mean increases
over time, that might indicate a shift in the population toward
conservatism. The standard deviation measures the dispersion of views in
the population -- if it increases over time, that might indicate an
increase in polarization.
\index{polarization, political}
\index{political polarization}

Compute the mean and standard deviation of
\passthrough{\lstinline!polviews74!} and
\passthrough{\lstinline!polviews22!}. What do they indicate about
changes over this interval?

\section{Plotting a Time Series}\label{plotting-a-time-series}

At this point we have looked at the endpoints, 1974 and 2022, but we
don't know what happened in between. To see how the distribution changed
over time, we can use \passthrough{\lstinline!groupby!} to group the
respondents by year.
\index{time series}
\index{groupby (Pandas method)}


\begin{lstlisting}[language=Python,style=source]
gss_by_year = gss.groupby('year')
type(gss_by_year)
\end{lstlisting}

\begin{lstlisting}[style=output]
pandas.core.groupby.generic.DataFrameGroupBy
\end{lstlisting}

The result is a \passthrough{\lstinline!DataFrameGroupBy!} object that
represents a collection of groups.

Now we can use the bracket operator to select the
\passthrough{\lstinline!polviews!} column.
\index{bracket operator}

\begin{lstlisting}[language=Python,style=source]
polviews_by_year = gss_by_year['polviews']
type(polviews_by_year)
\end{lstlisting}

\begin{lstlisting}[style=output]
pandas.core.groupby.generic.SeriesGroupBy
\end{lstlisting}

A column from a \passthrough{\lstinline!DataFrameGroupBy!} is a
\passthrough{\lstinline!SeriesGroupBy!}. If we invoke
\passthrough{\lstinline!mean!} on it, the result is a series that
contains the mean of \passthrough{\lstinline!polviews!} for each year of
the survey.

\begin{lstlisting}[language=Python,style=source]
mean_series = polviews_by_year.mean()
\end{lstlisting}

And here's what it looks like.

\begin{lstlisting}[language=Python,style=source]
mean_series.plot(color='C2', label='polviews')
decorate(xlabel='Year',
         ylabel='Mean (7 point scale)',
         title='Mean of polviews')
\end{lstlisting}

\begin{center}
\includegraphics[scale=0.6666666]{14_polviews_files/14_polviews_52_0.png}
\end{center}

The mean increased between 1974 and 2000, decreased since then, and
ended up almost where it started.

\pagebreak

The difference between the highest and
lowest points is only 0.34 points on a 7-point scale, so none of these
changes are drastic.

\begin{lstlisting}[language=Python,style=source]
mean_series.max() - mean_series.min()
\end{lstlisting}

\begin{lstlisting}[style=output]
0.34240143126104083
\end{lstlisting}

\textbf{Exercise:} The standard deviation quantifies the spread of the
distribution, which is one way to measure polarization. Plot standard
deviation of \passthrough{\lstinline!polviews!} for each year of the
survey from 1972 to 2022. Does it show evidence of increasing
polarization?
\index{polarization, political}
\index{political polarization}

\section{Smoothing the Curve}\label{smoothing-the-curve}

In the previous section we plotted mean and standard deviation of
\passthrough{\lstinline!polviews!} over time. In both plots, the values
are highly variable from year to year. We can use \textbf{local
regression} to compute a smooth line through these data points.
\index{smoothing}
\index{lowess (StatsModels function)}
\index{locally weighted scatterplot smoothing}

The following function takes a Pandas \passthrough{\lstinline!Series!}
and uses an algorithm called LOWESS to compute a smooth line. LOWESS
stands for ``locally weighted scatterplot smoothing''.

\begin{lstlisting}[language=Python,style=source]
from statsmodels.nonparametric.smoothers_lowess import lowess

def make_lowess(series):
    '''Use LOWESS to compute a smooth line.

    series: pd.Series

    returns: pd.Series
    '''
    y = series.values
    x = series.index.values

    smooth = lowess(y, x)
    index, data = np.transpose(smooth)

    return pd.Series(data, index=index)
\end{lstlisting}

\pagebreak

We'll use the following function to plot data points and the smoothed
line.

\begin{lstlisting}[language=Python,style=source]
def plot_series_lowess(series, color):
    '''Plots a series of data points and a smooth line.

    series: pd.Series
    color: string or tuple
    '''
    series.plot(linewidth=0, marker='o', color=color, alpha=0.5)
    smooth = make_lowess(series)
    smooth.plot(label='', color=color)
\end{lstlisting}

The following figure shows the mean of
\passthrough{\lstinline!polviews!} and a smooth line.

\begin{lstlisting}[language=Python,style=source]
mean_series = gss_by_year['polviews'].mean()
plot_series_lowess(mean_series, 'C2')
decorate(xlabel='Year',
         ylabel='Mean (7 point scale)',
         title='Mean of polviews')
\end{lstlisting}

\begin{center}
\includegraphics[scale=0.6666666]{14_polviews_files/14_polviews_61_0.png}
\end{center}

One reason the PMFs for 1974 and 2022 did not look very different is
that the mean went up (more conservative) and then down again (more
liberal). Generally, it looks like the U.S. has been trending toward
liberal for the last 20 years, or more, at least in the sense of how
people describe themselves.

\textbf{Exercise:} Use \passthrough{\lstinline!plot\_series\_lowess!} to
plot the standard deviation of \passthrough{\lstinline!polviews!} with a
smooth line.

\pagebreak

\section{Cross Tabulation}\label{cross-tabulation}

In the previous sections, we treated \passthrough{\lstinline!polviews!}
as a numerical quantity, so we were able to compute means and standard
deviations. But the responses are really categorical, which means that
each value represents a discrete category, like `liberal' or
`conservative'. In this section, we'll treat
\passthrough{\lstinline!polviews!} as a categorical variable, compute
the number of respondents in each category during each year, and plot
changes over time.

Pandas provides a function called \passthrough{\lstinline!crosstab!}
that computes a \textbf{cross tabulation} Here's how we can use to
compute the number of respondents in each category during each year.
\index{cross tabulation}
\index{crosstab (Pandas function)}

\begin{lstlisting}[language=Python,style=source]
year = gss['year']
column = gss['polviews']

xtab = pd.crosstab(year, column)
\end{lstlisting}

The result is a \passthrough{\lstinline!DataFrame!} with one row for
each year and one column for each category. Here are the first few rows.

\begin{lstlisting}[language=Python,style=source]
xtab.head()
\end{lstlisting}

\begin{tabular}{lrrrrrrr}
\toprule
polviews & 1 & 2 & 3 & 4 & 5 & 6 & 7 \\
year &  &  &  &  &  &  &  \\
\midrule
1974 & 31 & 201 & 211 & 538 & 223 & 181 & 30 \\
1975 & 56 & 184 & 207 & 540 & 204 & 162 & 45 \\
1976 & 31 & 198 & 175 & 564 & 209 & 206 & 34 \\
1977 & 37 & 181 & 214 & 594 & 243 & 164 & 42 \\
1978 & 21 & 140 & 255 & 559 & 265 & 187 & 25 \\
\bottomrule
\end{tabular}

It contains one row for each value of \passthrough{\lstinline!year!} and
one column for each value of \passthrough{\lstinline!polviews!}. Reading
the first row, we see that in 1974, 31 people gave response
\passthrough{\lstinline!1!}, 201 people gave response
\passthrough{\lstinline!2!}, and so on.

The number of respondents varies from year to year. To make meaningful
comparisons over time, we need to normalize the results, which means
computing for each year the \emph{fraction} of respondents in each
category, rather than the count. \passthrough{\lstinline!crosstab!}
takes an optional argument that normalizes each row.
\index{normalize}

\begin{lstlisting}[language=Python,style=source]
xtab_norm = pd.crosstab(year, column, normalize='index')
\end{lstlisting}

Here's what that looks like for the 7-point scale.

\begin{lstlisting}[language=Python,style=source]
xtab_norm.head()
\end{lstlisting}

\begin{tabular}{lrrrrrrr}
\toprule
polviews & 1 & 2 & 3 & 4 & 5 & 6 & 7 \\
year &  &  &  &  &  &  &  \\
\midrule
1974 & 0.0219081 & 0.142049 & 0.149117 & 0.380212 & 0.157597 & 0.127915 & 0.0212014 \\
1975 & 0.0400572 & 0.131617 & 0.148069 & 0.386266 & 0.145923 & 0.11588 & 0.0321888 \\
1976 & 0.0218772 & 0.139732 & 0.1235 & 0.398024 & 0.147495 & 0.145378 & 0.0239944 \\
1977 & 0.0250847 & 0.122712 & 0.145085 & 0.402712 & 0.164746 & 0.111186 & 0.0284746 \\
1978 & 0.0144628 & 0.0964187 & 0.17562 & 0.384986 & 0.182507 & 0.128788 & 0.0172176 \\
\bottomrule
\end{tabular}

Looking at the numbers in the table, it's hard to see what's going on.
In the next section, we'll plot the results.

To make the results easier to interpret, I'm going to replace the
numeric codes 1-7 with strings. First I'll make a dictionary that maps
from numbers to strings:

\begin{lstlisting}[language=Python,style=source]
polviews_map = {
    1: 'Extremely liberal',
    2: 'Liberal',
    3: 'Slightly liberal',
    4: 'Moderate',
    5: 'Slightly conservative',
    6: 'Conservative',
    7: 'Extremely conservative',
}
\end{lstlisting}

Then we can use the \passthrough{\lstinline!replace!} function like
this:
\index{replace (Pandas method)}

\begin{lstlisting}[language=Python,style=source]
polviews7 = gss['polviews'].replace(polviews_map)
\end{lstlisting}

We can use \passthrough{\lstinline!values!} to confirm that the values
in \passthrough{\lstinline!polviews7!} are strings.

\begin{lstlisting}[language=Python,style=source]
values(polviews7)
\end{lstlisting}

\begin{lstlisting}[style=output]
polviews
Conservative               9612
Extremely conservative     2145
Extremely liberal          2095
Liberal                    7309
Moderate                  24157
Slightly conservative      9816
Slightly liberal           7799
NaN                        9457
Name: count, dtype: int64
\end{lstlisting}

\pagebreak

If we make the cross tabulation again, we can see that the column names
are strings.

\begin{lstlisting}[language=Python,style=source]
xtab_norm = pd.crosstab(year, polviews7, normalize='index')
xtab_norm.columns
\end{lstlisting}

\begin{lstlisting}[style=output]
Index(['Conservative', 'Extremely conservative', 'Extremely liberal',
       'Liberal', 'Moderate', 'Slightly conservative', 'Slightly liberal'],
      dtype='object', name='polviews')
\end{lstlisting}

We are almost ready to plot the results, but first we need some colors.

\section{Color Palettes}\label{color-palettes}

To represent political views, we'll use a color palette from blue to
purple to red. Seaborn provides a variety of color palettes -- we'll
start with this one, which includes shades of blue and red. To represent
moderates, we'll replace the middle color with purple.
\index{color palettes}
\index{Seaborn}

\begin{lstlisting}[language=Python,style=source]
palette = sns.color_palette('RdBu_r', 7)
palette[3] = 'purple'
sns.palplot(palette)
\end{lstlisting}

\begin{center}
\includegraphics[scale=0.6666666]{14_polviews_files/14_polviews_84_0.png}
\end{center}

We'll make a dictionary that maps from the responses to the
corresponding colors.

\begin{lstlisting}[language=Python,style=source]
color_map = {}

for i, group in polviews_map.items():
    color_map[group] = palette[i-1]
\end{lstlisting}

Now we're ready to plot.

\section{Plotting a Cross Tabulation}\label{plotting-a-cross-tabulation}

To see how the fraction of people with each political alignment has
changed over time, we'll use
\passthrough{\lstinline!plot\_series\_lowess!} to plot the columns from
\passthrough{\lstinline!xtab\_norm!}.

\pagebreak

Here are the seven categories
plotted as a function of time. The
\passthrough{\lstinline!bbox\_to\_anchor!} argument passed to
\passthrough{\lstinline!plt.legend!} puts the legend outside the axes of
the figure.

\begin{lstlisting}[language=Python,style=source]
groups = list(polviews_map.values())

for group in groups:
    series = xtab_norm[group]
    plot_series_lowess(series, color_map[group])

decorate(
    xlabel='Year',
    ylabel='Proportion',
    title='Fraction of respondents with each political view',
)

plt.legend(bbox_to_anchor=(1.02, 1.02));
\end{lstlisting}

\begin{center}
\includegraphics[scale=0.6666666]{14_polviews_files/14_polviews_90_0.png}
\end{center}

This way of looking at the results suggests that changes in political
alignment during this period have generally been slow and small.

The fraction of self-described moderates has decreased slightly.
The fraction of conservatives increased, but seems to be decreasing now -- and the fraction of liberals seems to be increasing.
The fraction of people at the extremes has increased, but it is hard to
see clearly in this figure.

\pagebreak

We can get a better view by plotting just
the extremes.

\begin{lstlisting}[language=Python,style=source]
selected_groups = ['Extremely liberal', 'Extremely conservative']

for group in selected_groups:
    series = xtab_norm[group]
    plot_series_lowess(series, color_map[group])

decorate(
    xlabel='Year',
    ylabel='Proportion',
    ylim=[0, 0.065],
    title='Fraction of respondents with extreme political views',
)
\end{lstlisting}

\begin{center}
\includegraphics[scale=0.6666666]{14_polviews_files/14_polviews_92_0.png}
\end{center}

I used \passthrough{\lstinline!ylim!} to set the limits of the y-axis so
it starts at zero, to avoid making the changes seem bigger than they
are. This figure shows that the fraction of people who describe
themselves as `extreme' has increased from about 2.5\% to about 5\%. In
relative terms, that's a big increase. But in absolute terms these tails
of the distribution are still small.

\textbf{Exercise:} Let's do a similar analysis with
\passthrough{\lstinline!partyid!}, which encodes responses to the
question:
\index{party affiliation}
\index{Democrat}
\index{Republican}
\index{Independent (political affiliation)}

\begin{quote}
Generally speaking, do you usually think of yourself as a Republican,
Democrat, Independent, or what?
\end{quote}

\pagebreak

The valid responses are:

\begin{longtable}[]{ll}
\toprule
Code & Response \\
\midrule
\endhead
\bottomrule
\endlastfoot
0 & Strong democrat \\
1 & Not str democrat \\
2 & Ind, near dem \\
3 & Independent \\
4 & Ind, near rep \\
5 & Not str republican \\
6 & Strong republican \\
7 & Other party \\
\end{longtable}

In the notebook for this chapter, there are some suggestions to get you
started.

Here are the steps I suggest:

\begin{enumerate}
\def\labelenumi{\arabic{enumi})}
\item
  If you have not already saved this notebook, you should do that first.
  If you are running on Colab, select `Save a copy in Drive' from the
  File menu.
\item
  Now, before you modify this notebook, make \emph{another} copy and
  give it an appropriate name.
\item
  Search and replace \passthrough{\lstinline!polviews!} with
  \passthrough{\lstinline!partyid!} (use `Edit-\textgreater Find and
  replace').
\item
  Run the notebook from the beginning and see what other changes you
  have to make.
\end{enumerate}

What changes in party affiliation do you see over the last 50 years? Are
things going in the directions you expected?

\section{Summary}\label{summary}

This chapter introduces two new tools: local regression for computing a
smooth curve through noisy data, and cross tabulation for counting the
number of people, or fraction, in each group over time.

Now that we have a sense of how political alignment as changed, in the
next chapter we'll explore the relationship between political alignment
and other beliefs and attitudes.
